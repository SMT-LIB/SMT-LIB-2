% syntax-summary.tex

%!TEX root = main.tex


%===============================================================================
\chapter{Concrete Syntax} \label{app:concrete-syntax}
%===============================================================================
\thispagestyle{empty}

{\small

\begin{minipage}{.95\linewidth}

\subsection*{Predefined symbols}

These symbols have a predefined meaning in \thisversion.
Note that they are not reserved words.
For instance, 
they could also be used in principle as user-defined sort or function symbols 
in scripts.
\medskip

\ter{Bool} \ 
\ter{continued-execution} \ 
\ter{error} \ 
\ter{false} \ 
\ter{immediate-exit} \ 
\ter{incomplete} \ 
\ter{logic} \ 
%\ter{none} \ 
\ter{memout} \ 
\ter{sat} \ 
\ter{success} \ 
\ter{theory} \ 
%\ter{timeout} \ 
\ter{true} \ 
\ter{unknown} \ 
\ter{unsupported} \ 
\ter{unsat}


\subsection*{Predefined keywords}

These keywords have a predefined meaning in \thisversion.
\medskip

\attr{all-statistics} \ 
\attr{assertion-stack-levels} \ 
\attr{authors} \ 
%\attr{axioms} \ 
\attr{category} \ 
\attr{chainable} \ 
\attr{definition} \ 
\attr{diagnostic-output-channel} \ 
\attr{error-behavior}
%\attr{expand-definitions} \ 
\attr{extensions} \ 
\attr{funs} \ 
\attr{funs-description} \ 
\attr{global-declarations} \ 
\attr{interactive-mode} \ 
\attr{language} \ 
\attr{left-assoc} \ 
\attr{license} \ 
\attr{name} \ 
\attr{named} \ 
\attr{notes} \ 
\attr{pattern} \ 
\attr{print-success} \ 
\attr{produce-assignments} \ 
\attr{produce-models} \ 
\attr{produce-proofs} \ 
\attr{produce-unsat-assumptions} \ 
\attr{produce-unsat-cores} \ 
\attr{random-seed} \ 
\attr{reason-unknown} \ 
\attr{regular-output-channel} \ 
\attr{reproducible-resource-limit} \ 
\attr{right-assoc} \ 
\attr{smt-lib-version} \ 
\attr{sorts} \ 
\attr{sorts-description} \ 
\attr{source} \ 
\attr{status} \ 
\attr{theories} \ 
\attr{values} \ 
\attr{verbosity} \ 
\attr{version}

\end{minipage}

%\newpage
\subsection*{Auxiliary Lexical Categories}

\aLexical

\newpage
\subsection*{Tokens}

\begin{description}

\item[Reserved Words]
\ 
\medskip

\begin{minipage}{.95\linewidth}
\textbf{General:} \ 
\ter{!} \ 
\verb|_| \ 
\ter{as} \ 
\ter{BINARY} \ 
\ter{DECIMAL} \ 
\ter{exists} \ 
\ter{HEXADECIMAL} \ 
\ter{forall} \ 
\ter{let} \ 
\ter{match} \ 
\ter{NUMERAL} \ 
\ter{par} \ 
\ter{STRING}
\medskip


\textbf{Command names:}
\ter{assert} \ 
\ter{check-sat} \ 
\ter{check-sat-assuming} \ 
\ter{declare-const} \ 
\ter{declare-datatype} \ 
\ter{declare-datatypes} \ 
\ter{declare-fun} \ 
\ter{declare-sort} \ 
\ter{\new{declare-sort-parameter}} \ 
\ter{\new{define-const}} \ 
\ter{define-fun} \ 
\ter{define-fun-rec} \ 
\ter{define-sort} \ 
\ter{echo} \ 
\ter{exit} \ 
\ter{get-assertions} \ 
\ter{get-assignment} \ 
\ter{get-info} \ 
\ter{get-model} \ 
\ter{get-option} \ 
\ter{get-proof} \ 
\ter{get-unsat-assumptions} \ 
\ter{get-unsat-core} \ 
\ter{get-value} \ 
\ter{pop} \ 
\ter{push} \ 
\ter{reset} \ 
\ter{reset-assertions} \ 
\ter{set-info} \ 
\ter{set-logic} \ 
\ter{set-option}
\end{minipage}
\medskip


\item[Other tokens]
\ 
\medskip

\ \ter{(} 

\ \ter{)}

\tokens
\medskip

\noindent
Members of the \nter{symbol} category starting with the character
\ \ter{@} \ or \ \ter{.} \  are reserved for solver use.
Solvers can use them respectively as identifiers for abstract values and
solver generated function symbols other than abstract values.
\end{description}


\subsection*{S-expressions}
\ 

\sexpressions


\subsection*{Identifiers}
\ 

\cIdentifiers

\newpage
\subsection*{Sorts}
\ 

\cSorts
\medskip


\subsection*{Attributes}
\ 

\cAttributes
\medskip


\subsection*{Terms}
\ 

\cTerms

\newpage
\subsection*{Theories}
\ 

\cTheories
\medskip

\subsection*{Logics}
\ 

\cLogics
\medskip

\newpage
\subsection*{Info flags}
\ 

\cInfoFlags
\medskip

\subsection*{Command options}
\ 

\cCommandOptions

\newpage
\subsection*{Commands}
\ 

\cCommands

\newpage
\subsection*{Command responses}
\ 

\cResponsesI

\cResponsesII
}



%===============================================================================
\chapter{Abstract Syntax}\label{app:abstract-syntax}
%===============================================================================
\thispagestyle{empty}

{\small

\subsection*{Common Notation}
\smallskip

% as removed for now:
%\trem{revise}

\[
\begin{array}{llll}
b \in \bo, & \text{the set of boolean values} \\
%r \in \ra, & \text{the set of non-negative rational numbers} \\
n \in \na, & \text{the set of natural numbers}  &
w \in \st, & \text{the set of character strings} \\
s \in \so, & \text{the set of sort symbols} &
u \in \sop, & \text{the set of sort parameters} \\
f \in \fu, & \text{the set of function symbols\ \ } &
x \in \va, & \text{the set of variables} \\
%a \in \at, &  \text{the set of attribute names\ \ } &
%v \in \val, &  \text{the set of attribute values} \\
T \in \te, & \text{the set of theory names} &
L \in \lo, & \text{the set of logic names}
\end{array}
\]

\subsection*{Sorts}
\ 

\sortterms


\subsection*{Terms}
\ 

\terms


\newpage
\subsection*{Well-sorting rules for terms}

\new{
$f{:}\tau_1\cdots\tau_n\tau \in \Sigma$
iff $f$ has some rank $\hat\tau_1\cdots\hat\tau_n\hat\tau$ in $\Sigma$, and
$\tau_1\cdots\tau_n\tau$ is an instance of $\hat\tau_1\cdots\hat\tau_n\hat\tau$
}
\medskip

\termrules
\medskip

\subsection*{Theories}
\ 

\theories
\medskip

\subsection*{Logics}
\ 

\logics
\medskip

%\subsection*{Command options and info names}
%\ 
%
%\commandOptions
%
%\newpage
%\subsection*{Commands}
%\ 
%
%\commands
%\newpage
%
%\subsection*{Command responses}
%\ 
%
%\responses

}

