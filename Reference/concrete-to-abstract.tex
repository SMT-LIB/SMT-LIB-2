% concrete-to-abstract.tex

%!TEX root = reference-manual.tex


%===============================================================================
\chapter{Concrete to Abstract Syntax} \label{app:concrete-abstract}
%===============================================================================
\thispagestyle{empty}


[To be provided in a later release]

%\begin{comment}

\newcommand{\tr}[1]{\ulcorner#1\urcorner}

\ctrem{To be revised and completed}
This appendix defines a translation from the concrete syntax
of the SMT-LIB language to the abstract syntax used in 
Part~\ref{part:semantics} of this document.
To do that we assume as given the following mappings:

\begin{itemize}
\item
$a_{\so}: \nter{identifier} \to \so$ such that 
$a_{\so}(\ter{Bool}) = \bool$,
\item
$a_{\sop}: \nter{symbol} \to \sop$,
\item
$a_{\va}: \nter{symbol} \to \va$,
\item
$a_{\fu}: \nter{spec\_constant} \cup \nter{identifier} \to \fu$
such that 
$a_{\fu}(\ter{not}) = \lnot$ and
$a_{\fu}(\ter{and}) = \land$,
\item
$a_{\at}: \nter{keyword} \to \at$,
\item
$a_{\val}: \nter{attribute\_value} \to \val$,
\item
$a_{\lo}: \nter{symbol} \to \lo$,
\item
$a_{\st}: \nter{string} \to \st$,
mapping each string literal to the string it contains,
\item
$a_{\na}: \nter{numeral} \to \na$,
mapping each numeral to its corresponding natural number as expected,
\item
$a_{\ra}: \nter{decimal} \to \ra$,
mapping each decimal to its corresponding rational number as expected.
\end{itemize}
%\bigskip

\begin{remark}
Some of these mappings have overlapping domains.
The reason is that certain concrete syntactic categories 
(such as, \nter{symbol} or \nter{numeral})
play different roles in the language.
Also, for generality, the mappings are not assumed to be injective.
Some (such as $a_{\na}$) definitely are, 
some (such as $a_{\ra}$) definitely are not\footnote{
Observe that $a_\ra(\ter{0.0}) = a_\ra(\ter{0.00})$.
},
some other may or not, depending on specific cases.\footnote{
For instance, 
in a logic of arithmetic modulo 4 allowing all numerals
as constant symbols,
it might be convenient for $a_{\fu}$ to map its
\nter{numeral} subdomain onto the set $\{0,1,2,3\}$.
}
The mappings $a_{\at}$ and $a_{\val}$ will be used 
to provide an abstraction for user-defined attributes.
In contrast, predefined attributes will be translated directly.
\qed
\end{remark}


\section*{Abstraction Function}

With the mappings above, the translation can be given formally
as an abstraction function $\tr{\_}$ from concrete syntax expressions $e$
to abstract syntax ones.
The function is parametrized by a mapping from 
 is defined by induction on the concrete syntax rules as follows.\bigskip

\noindent \nter{sort} \\

\(
\begin{array}{llll}
 \tr{e}_x & := & a_{\so}(e)
 & \text{if } e \in \nter{identifier}
 \\[1ex]
 \tr{\ter{(}e\ e_1\ \cdots\ e_n \ter{)}}  & := & s\ \sigma_1\ \cdots \sigma_n &
 \text{with } \begin{cases} s = a_{\so}(e)\\ \sigma_i = \tr{e_i}\end{cases}
\end{array}
\)
\bigskip

\noindent \nter{attribute} \\

\(
\begin{array}{llll}
 \tr{e} & := & a_{\at}(e)
 & \text{if } e \in \nter{keyword}
 \\[1ex]
 \tr{e_1\ e_2}  & := & a\ v &
 \text{with } \begin{cases} a = a_{\at}(e_1)\\ v = a_{\val}(e_2)\end{cases}
\end{array}
\)
\bigskip

\noindent \nter{qual\_identifier} \\

\(
\begin{array}{llll}
 \tr{e} & := & a_{\fu}(e)
 & \text{if } e \in \nter{identifier}
 \\[1ex]
 \tr{\ter{(as } e_1\ e_2 \ter{)}}  & := & f^\sigma &
 \text{with } \begin{cases}f = a_{\fu}(e_1)\\ \sigma = \tr{e_2}\end{cases}
\end{array}
\)
\bigskip

\noindent \nter{var\_binding}

\(
\begin{array}{llll}
 \tr{\ter{(} e_1\ e_2 \ter{)}}  & := & x = t
 & \text{with } \begin{cases}x = a_{\va}(e)\\ t = \tr{e_2}\end{cases}
\end{array}
\)
\bigskip

\noindent \nter{sorted\_var}

\(
\begin{array}{llll}
 \tr{\ter{(} e_1\ e_2 \ter{)}}  & := & x{:}\sigma
 & \text{with } \begin{cases} x = a_{\va}(e)\\ \sigma = \tr{e_2}\end{cases}
\end{array}
\)
\bigskip

\newpage
\noindent 
\nter{term}\\

\(
\begin{array}{llll}
 \tr{e} & := & a_{\fu}(e)
 & \text{if } e \in \nter{spec\_constant}
 \\[1ex]
 \tr{\ter{(= } e_1\ e_2 \ter{)}} & := & 
 \tr{e_1}\; \eqs\; \tr{e_2}
 \\[1ex]
 \tr{\ter{(= } e_1\ \cdots\ e_n \ter{)}} & := & 
 \tr{\ter{(= } e_1\ e_2 \ter{)}} \;\land\;
 \tr{\ter{(= } e_2\ \cdots\ e_n \ter{)}}
 & \text{if } n > 2
 \\[1ex]
 \tr{\ter{(distinct } e_1\ e_2 \ter{)}} & := & 
 \lnot\ (\tr{e_1}\; \eqs\ \tr{e_2})
 \\[1ex]
 \tr{\ter{(distinct } e_1\ \cdots\ e_n \ter{)}} & := & 
 \bigwedge \left(\begin{array}{c}
             \lnot\ (\tr{e_1}\; \eqs\; \tr{e_2}) \\
             \vdots \\
             \ \lnot\ (\tr{e_1}\; \eqs\; \tr{e_n}) \\
             \tr{\ter{(distinct } e_2\ \cdots\ e_n \ter{)}}
             \end{array}
             \right)
 & \text{if } n > 2
 \\[7ex]
 \tr{\ter{(let (} e_1\ \cdots e_n\ \ter{) } e \ter{)}} & := & 
 \akey{let}\ \tr{e_1}\ \cdots\ \tr{e_n}\ \akey{in}\ \tr{e}
 \\[1ex]
 \tr{\ter{(forall (} e_1\ \cdots\ e_n \ter{) } e \ter{)}} & := & 
 \forall\; (\tr{e_1}\ \cdots\ \tr{e_n})\ \tr{e}
 \\[1ex]
 \tr{\ter{(exits (} e_1\ \cdots\ e_n \ter{) } e \ter{)}} & := & 
 \exists\; (\tr{e_1}\ \cdots\ \tr{e_n})\ \tr{e}
 \\[1ex]
 \tr{\ter{(!}\ e\ e_1\ \cdots\ e_n \ter{)}} & := & 
 t\ \alpha_1\ \cdots\ \alpha_n
 & \text{with } \begin{cases} t = \tr{e}\\ \alpha_i = \tr{e_i}\end{cases}
\end{array}
\)
\bigskip

\noindent 
In the last definition above,
$\eqs$ and $\land$ are used as infix symbols,
as customary.
Furthermore,
$\bigwedge\: (t_1 \ \cdots \ t_n)$ abbreviates
$t_1\; \land \; \bigwedge\: (t_2\ \cdots \ t_n)$
for $n>2$ and stands for $t_1 \land t_2$ for $n=2$. 
\bigskip


\noindent \nter{option} \\

\(
\begin{array}{llll}
 \tr{\attr{print-success}\ e} & := & \akey{print-success}\ b &
 \text{with } b = \tr{e}
 \\[1ex] 
 \tr{\attr{expand-definitions}\ e} & := & \akey{expand-definitions}\ b &
 \text{with } b = \tr{e}
 \\[1ex] 
 \tr{\attr{interactive-mode}\ e} & := & \akey{interactive-mode}\ b &
 \text{with } b = \tr{e}
 \\[1ex] 
 \tr{\attr{timeout}\ \ter{none}} & := & \akey{timeout}\ \akey{none}
 \\ 
 \tr{\attr{timeout}\ e} & := & \akey{timeout}\ r
 & \text{if } \begin{cases} e \in \nter{decimal}\\ r = a_{\ra}(e)\end{cases}
 \\[3ex] 
 \tr{\attr{regular-output-channel}\ e} & := & \akey{regular-output-channel}\ w
 & \text{with } w = a_{\st}(e)
 \\[1ex] 
 \tr{\attr{diagnostic-output-channel}\ e} & := & \akey{diagnostic-output-channel}\ w
 & \text{with } w = a_{\st}(e)
 \\[1ex] 
 \tr{\attr{random-seed}\ e} & := & \akey{random-seed}\ n
 & \text{with } n = a_{\na}(e)
 \\[1ex] 
 \tr{\attr{verbosity}\ e} & := & \akey{verbosity}\ n
 & \text{with } n = a_{\na}(e)
 \\[1ex] 
 \tr{e_1\ e_2} & := & a\ v
 & \text{with } \begin{cases} a = a_{\at}(e_1)\\ v = a_{\val}(e_2)\end{cases}
 \\[1ex] 
\end{array}
\)
\bigskip

\noindent \nter{bin\_value} \\

\(
\begin{array}{llll}
 \tr{\ter{true}} & := & \akey{true} \\
 \tr{\ter{false}} & := & \akey{false} 
\end{array}
\)
\bigskip

\noindent \nter{command} \\

\(
\begin{array}{llll}
 \tr{\ter{(set-logic}\ e \ter{)}} & := & \akey{set-logic}\ L
 & \text{with } L = a_{\lo}(e)
 \\[1ex] 
 \tr{\ter{(set-option}\ e \ter{)}} & := & \akey{set-option}\ \tr{e}
 \\[1ex] 
 \tr{\ter{(set-info}\ e \ter{)}} & := & \akey{set-info}\ \tr{e}
 \\ 
 \tr{\ter{(declare-sort}\ e_1\ e_2 \ter{)}} & := & \akey{declare-sort}\ s\ n
 & \text{with } \begin{cases} s = a_{\so}(e_1)\\ n = a_{\na}(e_2)\end{cases}
 \\[3ex] 
 \tr{\ter{(define-sort}\ e\ \ter{(}e_1\ \cdots\ e_n \ter{)}\ e'\ter{)}} & := & \akey{define-sort}\ s\ u_1\ \cdots u_n\ \tau
 & \text{with } 
   \begin{cases}
    s = a_{\so}(e)\\
    u_i = a_{\sop}(e_i)\\
    \tau = \tr{e'}\\
   \end{cases}
 \\[4ex] 
 \tr{\ter{(push}\ e \ter{)}} & := & \akey{push}\ n
 & \text{with } n = a_{\na}(e)
 \\[1ex] 
 \tr{\ter{(pop}\ e \ter{)}} & := & \akey{pop}\ n
 & \text{with } n = a_{\na}(e)
 \\[1ex] 
 \tr{\ter{(assert}\ e \ter{)}} & := & \akey{assert}\ t
 & \text{with } t = \tr{e}
 \\[1ex] 
 \tr{\ter{(assert}\ e\ e' \ter{)}} & := & \akey{assert}\ t\ l
 & \text{with } \begin{cases}t = \tr{e}\\ l = a_{\la}(e')\end{cases}
 \\[1ex] 
 \tr{\ter{(check-sat)}} & := & \akey{check-sat}
 \\[1ex] 
 \tr{\ter{(get-assertions)}} & := & \akey{get-assertions}
 \\[1ex] 
 \tr{\ter{(check-sat)}} & := & \akey{check-sat}
 \\[1ex] 
 \tr{\ter{(get-model)}} & := & \akey{get-model}
 \\[1ex] 
 \tr{\ter{(get-value (} e_1\ \cdots\ e_n\ter{))}} & := & \akey{get-value}\ t_1\ \cdots\ t_n
  & \text{with } t_i = \tr{e_i}
 \\[1ex] 
 \tr{\ter{(get-proof)}} & := & \akey{get-proof}
 \\[1ex] 
 \tr{\ter{(get-unsat-core)}} & := & \akey{get-unsat-core}
 \\[1ex] 
 \tr{\ter{(get-info}\ e \ter{)}} & := & \akey{get-info}\ \tr{e}
 \\[1ex] 
 \tr{\ter{(get-option}\ e \ter{)}} & := & \akey{get-option}\ \tr{e}
 \\[1ex] 
 \tr{\ter{(exit)}} & := & \akey{exit}
\end{array}
\)
\bigskip

%\end{comment}
