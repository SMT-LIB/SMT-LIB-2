%!TEX root = main.tex

\newcommand{\alt}{\ensuremath{~\mid~}}
\newcommand{\akey}[1]{\textbf{#1}}
\newcommand{\seque}[1]{l_{\mathrm{#1}}}
\newcommand{\bool}{\akey{Bool}}
\newcommand{\fconstruct}{{\contour{black}{$\rightarrow$}}}
\newcommand{\fconstructa}[2]{#1\ \fconstruct\ #2}
\newcommand{\fapply}{\akey{apply}}
\newcommand{\fapplya}[2]{\akey{apply}(#1, #2)}

\newcommand{\nter}[2][]{{\textcolor{OliveGreen}{\ensuremath{\langle}\emph{\sffamily{#2}}\ensuremath{\rangle^{#1}}}}\xspace}

\newcommand{\expr}[1]{{\textcolor{NavyBlue}{\texttt{#1}}}}
\newcommand{\ter}[1]{\expr{#1}\xspace}
\newcommand{\attr}[1]{\expr{:#1}\xspace}
\newcommand{\kw}[1]{$\backslash$\ter{#1}}


\newcommand{\aattr}[1]{\textbf{#1}\xspace}

%\newcommand{\opar}{\text{\bf(}}
%\newcommand{\cpar}{\text{\bf)}}


%----------------------------------------
% Concrete syntax
%----------------------------------------

\newcommand{\aLexical}{
\begin{tabular}{lcl}
 \nter{white\_space\_char} & ::= &
 \dec{9} \alt\  
 \dec{10} \alt\  
 \dec{13} \alt\  
 \dec{32}
 \\[1ex]
 \nter{printable\_char} & ::= &
 \dec{32}  \alt\ $\cdots$ \alt\ \dec{126} \alt\ \dec{128} \alt\ $\cdots$ \alt\ \dec{255}
 \\[1ex]
 \nter{digit} & ::= & \ter{0} \alt\ $\cdots$ \alt\ \ter{9}
 \\[1ex]
 \nter{letter} & ::= &
 \ter{A}  \alt\ $\cdots$ \alt\ \ter{Z} \alt\ 
 \ter{a} \alt\ $\cdots$ \alt\ \ter{z}
\end{tabular}
}

\newcommand{\tokens}{
\begin{tabular}{lcl}
 \nter{numeral} & ::= & \ter{0}
                  \alt\ \textit{a non-empty sequence of digits not starting with}
                        \ter{0}
 \\[1ex]
 \nter{decimal} & ::= & \nter{numeral}\!\!\ter{.0}\!\!$^*$\nter{numeral}
 \\[1ex]
 \nter{hexadecimal} & ::= & \ter{\#x} \textit{followed by a non-empty sequence of digits and letters}\\
            &     & \textit{from} \ter{A} \textit{to} \ter{F}\textit{, capitalized or not}
 \\[1ex]
 \nter{binary} & ::= & \ter{\#b}\textit{followed by a non-empty sequence of} \ter{0} \textit{and} \ter{1} \textit{characters}
 \\[1ex]
 \nter{string} & ::= & \textit{sequence of whitespace and printable characters in double quotes 
                                } \\
               &     & \textit{with escape sequence \ter{""}}
 \\[1ex]
 \nter{simple\_symbol} & ::= & \textit{a non-empty sequence of letters, digits and
                               the characters} \\
               &     & \ter{+ - / * = \% ?~!~.~\$ \_ \~{} \& \^{} $<$ $>$ @}
                       \textit{that does not start} \\
               &     & \textit{with a digit}
 \\[1ex]
 \nter{symbol} & ::= & \nter{simple\_symbol} \\
               &\alt & \textit{a sequence of whitespace and printable characters that }\\
               &     & \textit{starts and ends with} \ter{|} \textit{and does not otherwise include} \ter{|} \textit{ or } $\backslash$
 \\[1ex]
 \nter{keyword} 
               & ::= & {\texttt{:}}\nter{simple\_symbol}
                        
\end{tabular}
}

\newcommand{\sexpressions}{
\begin{tabular}{lcl}
 \nter{spec\_constant} & ::= & \nter{numeral} \alt\ \nter{decimal}
                         \alt\ \nter{hexadecimal} \alt\ \nter{binary}
                         \alt\ \nter{string}
 \\[1ex]
 \nter{s\_expr} & ::= & \nter{spec\_constant} \alt\ \nter{symbol}
                  \alt\ \nter{reserved} \alt\ \nter{keyword} \\
                & \alt& \ter{(} \nter[*]{s\_expr} \ter{)}
\end{tabular}
}


\newcommand{\cIdentifiers}{
\begin{tabular}{lcl}
 \nter{index} & ::= & \nter{numeral} \alt\ \nter{symbol}
 \\[1ex]
 \nter{identifier} & ::= & \nter{symbol}
                     \alt\ \ter{(} \ter{\_} \nter{symbol} \nter[+]{index} \ter{)}
\end{tabular}
}

\newcommand{\cSorts}{
\begin{tabular}{lcl}
 \nter{sort} & ::= & \nter{identifier}
               \alt\ \ter{(} \nter{identifier} \nter[+]{sort} \ter{)}
\end{tabular}
}

\newcommand{\cAttributes}{
\begin{tabular}{lcl}
 \nter{attribute\_value} & ::= & \nter{spec\_constant} \alt\ \nter{symbol}
                  \alt\ \ter{(} \nter[*]{s\_expr} \ter{)}
 \\[1ex]
 \nter{attribute} & ::= & \nter{keyword}
                     \alt\ \nter{keyword} \nter{attribute\_value}
\end{tabular}
}

\newcommand{\cTerms}{
\begin{tabular}{lcl}
 \nter{qual\_identifier} & ::= & \nter{identifier}
                            \alt\ \ter{(} \ter{as} \nter{identifier} \nter{sort} \ter{)}
 \\[1ex]
% \nter{qual\_constructor} & ::= & \nter{symbol}
%                            \alt\ \ter{(} \ter{as} \nter{symbol} \nter{sort} \ter{)}
% \\[1ex]
 \nter{var\_binding} & ::= & \ter{(} \nter{symbol} \nter{term} \ter{)}
 \\[1ex]
 \nter{sorted\_var} & ::= & \ter{(} \nter{symbol} \nter{sort} \ter{)}
 \\[1ex]
 \nter{pattern} & ::= & \nter{symbol} % \\
%                & \alt\ & \nter{qual\_constructor} \\
%                & \alt\ & \ter{(} \nter{qual\_constructor} \nter[+]{symbol} \ter{)}
                 \alt\ \ter{(} \nter{symbol} \nter[+]{symbol} \ter{)}
 \\[1ex]
 \nter{match\_case} & ::= & \ter{(} \nter{pattern} \nter{term} \ter{)}
 \\[1ex]
 \nter{term} & ::= & \nter{spec\_constant} \\
             &\alt &  \nter{qual\_identifier} \\ 
             &\alt & \ter{(} \nter{qual\_identifier} \nter[+]{term} \ter{)} \\
             &\alt & \ter{(} \ter{let} 
                             \ter{(} \nter[+]{var\_binding} \ter{)} \nter{term}
                     \ter{)} \\
             &\alt & \ter{(} \ter{lambda} 
                             \ter{(} \nter[+]{sorted\_var} \ter{)} \nter{term}
                     \ter{)} \\
             &\alt & \ter{(} \ter{forall} 
                             \ter{(} \nter[+]{sorted\_var} \ter{)} \nter{term}
                     \ter{)} \\
             &\alt & \ter{(} \ter{exists} 
                             \ter{(} \nter[+]{sorted\_var} \ter{)} \nter{term}
                     \ter{)} \\
             &\alt & \ter{(} \ter{match} \nter{term}  
                             \ter{(} \nter[+]{match\_case} \ter{)}
                     \ter{)} \\
             &\alt & \ter{(} \ter{!} \nter{term} \nter[+]{attribute} \ter{)}
\end{tabular}
}


\newcommand{\cTheories}{
\begin{tabular}{lcl}
 \nter{sort\_symbol\_decl} 
  & ::= & \ter{(} \nter{identifier} \nter{numeral} \nter[*]{attribute} \ter{)}
 \\[1ex]
 \nter{meta\_spec\_constant}
  & ::= & \ter{NUMERAL} \alt\ \ter{DECIMAL} \alt\ \ter{STRING}
 \\[1ex]
 \nter{fun\_symbol\_decl}
  & ::= & \ter{(} \nter{spec\_constant} \nter{sort} \nter[*]{attribute} \ter{)}\\
  & \alt& \ter{(} \nter{meta\_spec\_constant} \nter{sort} \nter[*]{attribute} 
          \ter{)} \\
  & \alt& \ter{(} \nter{identifier} \nter[+]{sort} \nter[*]{attribute} \ter{)}
 \\[1ex]
 \nter{par\_fun\_symbol\_decl}
  & ::= & \nter{fun\_symbol\_decl}\\
  & \alt& \ter{(} \ter{par} \ter{(} \nter[+]{symbol} \ter{)} %\\ &    &  \ \ 
  \ter{(} \nter{identifier} \nter[+]{sort} \nter[*]{attribute} \ter{)} 
          \ter{)}
 \\[1ex]
 \nter{theory\_attribute}
  & ::=  & \attr{sorts} \ter{(} \nter[+]{sort\_symbol\_decl} \ter{)} \\
  & \alt & \attr{funs}\ \ter{(} \nter[+]{par\_fun\_symbol\_decl} \ter{)} \\
  & \alt & \attr{sorts-description}\ \nter{string}  \\
  & \alt & \attr{funs-description}\ \nter{string}  \\
  & \alt & \attr{definition}\ \nter{string}  \\
%ct  & \alt & \attr{axioms} \ter{(} \nter[+]{term} \ter{)} \\
  & \alt & \attr{values}\ \nter{string} \\
  & \alt  & \attr{notes}\ \nter{string}  \\
  & \alt & \nter{attribute}
 \\[1ex]
  \nter{theory\_decl}
 & ::= & \ter{(} \ter{theory}\ \nter{symbol} \nter[+]{theory\_attribute} \ter{)}
\end{tabular}
}


\newcommand{\cLogics}{
\begin{tabular}{lcl}
 \nter{logic\_attribute}
  & := & \attr{theories}\ \ter{(} \nter[+]{symbol} \ter{)} \\
  & \alt & \attr{language}\ \nter{string} \\
  & \alt & \attr{extensions}\ \nter{string} \\
  & \alt & \attr{values}\ \nter{string} \\
  & \alt & \attr{notes}\ \nter{string}  \\
  & \alt & \nter{attribute}
 \\[1ex]
  \nter{logic}
  & ::= & \ter{(} \ter{logic}\ \nter{symbol} \nter[+]{logic\_attribute} \ter{)}
\end{tabular}
}

\newcommand{\cCommandOptions}{
\begin{tabular}{lcl}
 \nter{b\_value}
  & ::= & \ter{true} \alt\ \ter{false}
 \\[2ex]
 \nter{option}
  & ::=  & \attr{diagnostic-output-channel} \nter{string} \\
%  & \alt & \attr{expand-definitions} \nter{b\_value} \\
  & \alt & \attr{global-declarations} \nter{b\_value} \\
  & \alt & \attr{interactive-mode} \nter{b\_value} \\
  & \alt & \attr{print-success} \nter{b\_value} \\
  & \alt & \attr{produce-assertions} \nter{b\_value} \\
  & \alt & \attr{produce-assignments} \nter{b\_value} \\
  & \alt & \attr{produce-models} \nter{b\_value} \\
  & \alt & \attr{produce-proofs} \nter{b\_value} \\
  & \alt & \attr{produce-unsat-assumptions} \nter{b\_value} \\
  & \alt & \attr{produce-unsat-cores} \nter{b\_value} \\
  & \alt & \attr{random-seed} \nter{numeral} \\
  & \alt & \attr{regular-output-channel} \nter{string} \\
  & \alt & \attr{reproducible-resource-limit} \nter{numeral} \\
  & \alt & \attr{verbosity} \nter{numeral} \\
  & \alt & \nter{attribute}
\end{tabular}
}

\newcommand{\cInfoFlags}{
\begin{tabular}{lcl}
 \nter{info\_flag}
  & ::=  & \attr{all-statistics} 
    \alt\  \attr{assertion-stack-levels} 
    \alt\  \attr{authors} \\
  & \alt & \attr{error-behavior}
    \alt\  \attr{name} 
    \alt\  \attr{reason-unknown} \\
%    \alt\  \attr{status} \\
  & \alt & \attr{version}
    \alt\  \nter{keyword}
\end{tabular}
}

\newcommand{\cCommands}{
\begin{tabular}{lcl}
 \nter{sort\_dec} & ::= & \ter{(} \nter{symbol} \nter{numeral} \ter{)}
 \\[1ex]
 \nter{selector\_dec} & ::= & \ter{(} \nter{symbol} \nter{sort} \ter{)}
 \\[1ex]
 \nter{constructor\_dec} & ::= & \ter{(} \nter{symbol} \nter[*]{selector\_dec} \ter{)}
 \\[1ex]
 \nter{datatype\_dec} & ::= & \ter{(} \nter[+]{constructor\_dec} \ter{)}
                       \alt  \ter{(} \ter{par} \ter{(} \nter[+]{symbol} \ter{)} \ter{(} \nter[+]{constructor\_dec} \ter{)} \ter{)}
 \\[1ex]
 \nter{function\_dec} & ::= & \ter{(} \nter{symbol} \ter{(} \nter[*]{sorted\_var} \ter{)} \nter{sort} \ter{)}
 \\[1ex]
 \nter{function\_def} & ::= & \nter{symbol} \ter{(} \nter[*]{sorted\_var} \ter{)} \nter{sort} \nter{term}
 \\[1ex]
 \nter{prop\_literal} & ::= & \nter{symbol} \alt\ \ter{(} \ter{not} \nter{symbol} \ter{)}
 \\[1ex]
 \nter{command}
  & ::=  & \ter{(} \ter{assert} \nter{term} \ter{)}\\
  & \alt & \ter{(} \ter{check-sat} \ter{)} \\
  & \alt & \ter{(} \ter{check-sat-assuming} \ter{(} \nter[*]{prop\_literal} \ter{)} \ter{)}
            \\
  & \alt & \ter{(} \ter{declare-const} \nter{symbol} \nter{sort} \ter{)}
            \\
  & \alt & \ter{(} \ter{declare-datatype} \nter{symbol} \nter{datatype\_dec}\ter{)}
            \\
  & \alt & \ter{(} \ter{declare-datatypes} \ter{(} \nter[n+1]{sort\_dec} \ter{)} \ter{(} \nter[n+1]{datatype\_dec} \ter{)} \ter{)}
            \\
  & \alt & \ter{(} \ter{declare-fun} \nter{symbol}
            \ter{(} \nter[*]{sort} \ter{)} \nter{sort}
           \ter{)} \\
  & \alt & \ter{(} \ter{declare-sort} \nter{symbol} \nter{numeral} \ter{)} \\
  & \alt & \ter{(} \ter{\new{declare-sort-parameter}} \nter{symbol} \ter{)} \\
  & \alt & \ter{(} \ter{\new{define-const}} \nter{symbol} \nter{sort} \nter{term} \ter{)} \\
  & \alt & \ter{(} \ter{define-fun} \nter{function\_def} \ter{)} \\
  & \alt & \ter{(} \ter{define-fun-rec} \nter{function\_def} \ter{)}
           \\
  & \alt & \ter{(} \ter{define-funs-rec} \ter{(} \nter[n+1]{function\_dec} \ter{)} \ter{(} \nter[n+1]{term} \ter{)} \ter{)}
           \\
  & \alt & \ter{(} \ter{define-sort} \nter{symbol}
            \ter{(} \nter[*]{symbol} \ter{)} \nter{sort}
           \ter{)} \\
  & \alt & \ter{(} \ter{echo} \nter{string} \ter{)}
           \\
  & \alt &  \ter{(} \ter{exit} \ter{)} \\
  & \alt & \ter{(} \ter{get-assertions} \ter{)} \\
  & \alt & \ter{(} \ter{get-assignment} \ter{)} \\
  & \alt & \ter{(} \ter{get-info} \nter{info\_flag} \ter{)} \\
  & \alt & \ter{(} \ter{get-model} \ter{)}
           \\
  & \alt & \ter{(} \ter{get-option} \nter{keyword} \ter{)} \\
  & \alt & \ter{(} \ter{get-proof} \ter{)}\\
  & \alt & \ter{(} \ter{get-unsat-assumptions}  \ter{)}
          \\
  & \alt & \ter{(} \ter{get-unsat-core}  \ter{)} \\
  & \alt & \ter{(} \ter{get-value} \ter{(} \nter[+]{term} \ter{)}
           \ter{)}
           \\
  & \alt & \ter{(} \ter{pop} \nter{numeral} \ter{)} \\
  & \alt & \ter{(} \ter{push} \nter{numeral} \ter{)} \\
  & \alt & \ter{(} \ter{reset} \ter{)}
           \\
  & \alt & \ter{(} \ter{reset-assertions} \ter{)}
           \\
  & \alt & \ter{(} \ter{set-info} \nter{attribute} \ter{)} \\
  & \alt & \ter{(} \ter{set-logic} \nter{symbol} \ter{)} \\
  & \alt & \ter{(} \ter{set-option} \nter{option} \ter{)}
 \\[1ex]
 \nter{script} & ::= & \nter[*]{command}
\end{tabular}
}


\newcommand{\cResponsesI}{
\begin{tabular}{lcl}
 \nter{error-behavior} 
  & ::= & \ter{immediate-exit} \alt \ter{continued-execution}
 \\[1ex]
 \nter{reason-unknown} 
  & ::= & \ter{memout} \alt \ter{incomplete} \alt \nter{s\_expr}
 \\[1ex]
 \nter{model\_response}
  & ::= & \ter{(} \ter{define-fun} \nter{function\_def} \ter{)} \alt
          \ter{(} \ter{define-fun-rec} \nter{function\_def} \ter{)}
          \\
  & \alt & \ter{(} \ter{define-funs-rec} \ter{(} \nter[n+1]{function\_dec} \ter{)} \ter{(} \nter[n+1]{term} \ter{)} \ter{)}
 \\[1ex]
 \nter{info\_response}
  & ::= & \attr{assertion-stack-levels} \nter{numeral} \\
  & \alt& \attr{authors} \nter{string}\\
  & \alt& \attr{error-behavior} \nter{error-behavior} \\
  & \alt& \attr{name} \nter{string}\\
  & \alt& \attr{reason-unknown} \nter{reason-unknown}\\
  & \alt& \attr{version} \nter{string}\\
  & \alt& \nter{attribute}
 \\[1ex]
 \nter{valuation\_pair} 
  & ::= & \ter{(} \nter{term} \nter{term} \ter{)}
 \\[1ex]
 \nter{t\_valuation\_pair} 
  & ::= & \ter{(} \nter{symbol} \nter{b\_value} \ter{)}
\end{tabular}
}

\newcommand{\cGeneralResponse}{
 \nter{general\_response}
   & ::=  & \ter{success}
     \alt\  \nter{specific\_success\_response} \\
   & \alt & \ter{unsupported}
     \alt\  \ter{(} \ter{error} \nter{string} \ter{)} 
}

\newcommand{\cResponsesII}{
\begin{tabular}{lcl}
 \nter{check\_sat\_response} & ::= & \ter{sat} \alt\ \ter{unsat} \alt\ \ter{unknown}
 \\[1ex]
 \nter{echo\_response} & ::= & \nter{string}
 \\[1ex]
 \nter{get\_assertions\_response} & ::= & \ter{(} \nter[*]{term} \ter{)}
 \\[1ex]
 \nter{get\_assignment\_response} & ::= & \ter{(} \nter[*]{t\_valuation\_pair} \ter{)}
 \\[1ex]
 \nter{get\_info\_response} & ::= & \ter{(} \nter[+]{info\_response} \ter{)}
 \\[1ex]
 \nter{get\_model\_response} & ::= & \ter{(} \nter[*]{model\_response} \ter{)}
 \\[1ex]
 \nter{get\_option\_response} & ::= & \nter{attribute\_value}
 \\[1ex]
 \nter{get\_proof\_response} & ::= & \nter{s\_expr}
 \\[1ex]
 \nter{get\_unsat\_assump\_response} & ::= & \ter{(} \nter[*]{symbol} \ter{)}
 \\[1ex]
 \nter{get\_unsat\_core\_response} & ::= & \ter{(} \nter[*]{symbol} \ter{)}
 \\[1ex]
 \nter{get\_value\_response} & ::= & \ter{(} \nter[+]{valuation\_pair} \ter{)}
 \\[1ex]
 \nter{specific\_success\_response}
   & ::= & \nter{check\_sat\_response}
     \alt\ \nter{echo\_response}
   \\
   & \alt & \nter{get\_assertions\_response}
     \alt\ \nter{get\_assignment\_response}
   \\
   & \alt & \nter{get\_info\_response}
     \alt\ \nter{get\_model\_response}
   \\
   & \alt & \nter{get\_option\_response}
     \alt\ \nter{get\_proof\_response}
   \\
   & \alt & \nter{get\_unsat\_assumptions\_response}
   \\
   & \alt & \nter{get\_unsat\_core\_response}
     \alt\  \nter{get\_value\_response}
 \\[2ex]
 \cGeneralResponse
\end{tabular}
}



%----------------------------------------
% Abstract syntax
%----------------------------------------

\newcommand{\sortterms}{
\(
\begin{array}{llcl}
 \text{(Monomorphic Sorts)} & 
 \sigma
 & ::= & s\: \sigma^*
 \\[2ex]
 \text{(Polymorphic Sorts)} &
 \tau & ::= & u \alt s\: \tau^*
\end{array}
\)
}

\begin{comment}
\[
\begin{array}{llll}
s \in \mathcal{S}, & \text{the set of sort symbols \ \ } &
u \in \mathcal{U}, & \text{the set of sort parameters} 
\end{array}
\]
\end{comment}


\newcommand{\terms}{
\(
\begin{array}{llcl}
%  \text{(Attributes)} &
%   \alpha & ::= & a \alt a = v
%  \\[2.2ex]
  \text{(Patterns)} &
  p & ::= & x \alt f\: x^*
  \\[2.2ex]
  \text{(Terms)} &
  \mathit{t} & ::= & x % \alt c
      \alt f\: t^* \alt f^\tau\: t^* \alt \lambda\, (x{:}\tau)\ t \alt
      \exists\, (x{:}\tau)\ t \alt
      \forall\, (x{:}\tau)\ t \\ %\alt t \eqs t \\
  & & \alt & 
  %& & \alt & 
  \akey{let}\ (x = t)^+\ \akey{in}\ t
      \alt   \akey{match}\ t\ \akey{with}\ (p \to t)^+
 \end{array}
\)
}

\begin{comment}
\[
\begin{array}{llll}
a \in \mathcal{A}, &  \text{the set of attribute names\ \ } &
v \in \mathcal{V}, &  \text{the set of attribute values} \\
%n \in \mathcal{N}, & \text{the set of numerals} &
%r \in \mathcal{R}, & \text{the set of positive rationals} \\
%h \in \mathcal{H}, & \text{the set of hexadecimal constants} &
%b \in \mathcal{BV}, & \text{the set of bitvector constants} \\
%w \in \mathcal{W}, & \text{the set of character strings}  &
x \in \mathcal{X}, & \text{the set of variables} &
f \in \mathcal{F}, & \text{the set of function symbols}
\end{array}
\]
\end{comment}



\newcommand{\termrules}{
%% variable
\begin{tabular}{ll}
  \derives
  {}
  {\Sigma \vdash x:\tau}
  &
  if $x{:}\tau \in \Sigma$
  \end{tabular}
  \bigskip

  %% function application
\begin{tabular}{ll}
  \derives{
  \Sigma \vdash t_1:\tau_1 \quad \cdots \quad
  \Sigma \vdash t_k:\tau_k
  }
  {\Sigma \vdash (f\; t_1\; \cdots\; t_k): \tau}
  &
  %\text{if $f:\tau_1 \cdots \tau_k\tau \in \Sigma$ and
  %         $f:\tau_1 \cdots \tau_k\tau' \notin \Sigma$ with 
  %         $\tau' \neq \tau$}
  if
  \(
  \begin{cases}
  f{:}\tau_1 \cdots \tau_k\tau \in \Sigma, \\
  f{:}\tau_1 \cdots \tau_k\tau' \notin \Sigma \ \text{ for all } \tau' \neq \tau
  \end{cases}
  \)
  \end{tabular}
  \bigskip
  
  %% lambda
\begin{tabular}{ll}
  \derives{
  \Sigma[x:\tau_1] \vdash t_2:\tau_2
  }
  {\Sigma \vdash (\lambda (x {:} \tau_1)\; t_2) : \fconstructa{\tau_1}{\tau_2}}
  &
  \end{tabular}
  \bigskip
  
  %% annotated function application
\begin{tabular}{ll}
  \derives{
  \Sigma \vdash t_1:\tau_1 \quad \cdots \quad
  \Sigma \vdash t_k:\tau_k
  }
  {\Sigma \vdash (f^\tau\; t_1\; \cdots\; t_k)\ : \tau}
  &
  if 
  $f{:}\tau_1 \cdots \tau_k\tau \in \Sigma$
  \end{tabular}
  \bigskip
  
%  %% %% application
% \begin{tabular}{ll}
%   \derives{
%   \Sigma \vdash t:\fconstructa{\tau_1}{\tau} \quad \Sigma \vdash t_1:\tau_1
%   }
%   {\Sigma \vdash \fapplya{t}{t_1}\ \alpha^*: \tau}
%   &
%   \end{tabular}
%   \bigskip
% There is no need for such a typing rule since \fapply is a normal interpreted
% function.  Typing rules above are sufficient.
  
  %% quantifiers
\begin{tabular}{ll}
\derives{\Sigma[x{:}\tau]\: \vdash t:\bool}
{\Sigma \vdash (Q x{:}\tau\; t)\ : \bool}
&
if $Q\, \in \{\exists, \forall\}$
\end{tabular}
\bigskip

 
  %% let
\begin{tabular}{ll}
\derives{
\Sigma \vdash t_1:\tau_1 \quad \cdots \quad
\Sigma \vdash t_{k+1}:\tau_{k+1} \\
\Sigma[x_1{:}\tau_1,\: \ldots,\: x_{k+1}{:}\tau_{k+1}]\: \vdash t:\tau}
{\Sigma \vdash (\akey{let}\ x_1 = t_1\;\cdots\; x_{k+1} = t_{k+1}\ \akey{in}\ t)\ : \tau}
&
if $x_1, \ldots, x_{k+1}$ are all distinct
\end{tabular}
\bigskip

%% match 1
\begin{tabular}{ll}
\derives{
\Sigma \vdash t:\delta \\
\Sigma[\bar x_i{:}\bar\tau_i]\: \vdash t_i:\tau \text{ for } i=1,\ldots,k+1
}
{\Sigma \vdash (\akey{match}\ t\ \akey{with}\ c_1\:\bar x_1 \to t_1 \;\cdots\; c_{k+1}\:\bar x_{k+1} \to t_{k+1})\ : \tau}
&
if 
\(
\begin{cases}
\{c_1,\ldots,c_{k+1}\} = \mathrm{con}_\Sigma(\delta), \\
\text{ for all } i=1,\ldots,k+1 \\
\quad c_i{:}\bar\sigma_i\delta \in \Sigma \text{ and}\\
\quad \bar{x}_i \text{ contains no repetitions}
\end{cases}
\)
\end{tabular}
\bigskip

%% match 2
\begin{tabular}{ll}
\derives{
\Sigma \vdash t:\delta \\
\Sigma[\bar x_i{:}\bar\tau_i]\: \vdash t_i:\tau \text{ for } i=1,\ldots,k\\
\Sigma[x_i{:}\delta]\: \vdash t_i:\tau \text{ for } i=k+1,\ldots,n 
}
{\Sigma \vdash (\akey{match}\ t\ \akey{with}\ p_1 \to t_1 \;\cdots\; p_n \to t_n)\ : \tau}
&
if 
\(
\begin{cases}
\{c_1,\ldots,c_k\} \subseteq \mathrm{con}_\Sigma(\delta) \neq \emptyset,  \\
\{p_1,\ldots, p_n\} = \{c_1\:\bar x_1, \ldots, c_k\:\bar x_k\} \cup {} \\
                      \hspace{6.7em} \{x_{k+1}, \ldots, x_n\}, \\
\;k < n,\\
\text{ for all } i=1,\ldots,k \\
\quad c_i{:}\bar\tau_i\delta \in \Sigma \text{ and}\\
\quad \bar{x}_i \text{ contains no repetitions}
\end{cases}
\)
\end{tabular}
\bigskip
}



\newcommand{\theories}{
\(
\begin{array}{llcl}
 \text{(Sort symbol declarations)} &
 \mathit{sdec} & ::= & s\: n
 \\[2ex]
 \text{(Fun.~symbol declarations)} &
 \mathit{fdec} & ::= & f\: \tau^+
 \\[2ex]
 \text{(Theory attributes)} &
 \mathit{tattr} & ::= & \aattr{sorts} = \mathit{sdec}^+
        \alt \aattr{funs} = \mathit{pdec}^+ \\
   & & \alt& \aattr{sorts-description} = w \\
   & & \alt& \aattr{funs-description} = w \\
   & & \alt& \aattr{definition} = w
       \alt  \aattr{values} = t^+% \\
   %& & \alt& \aattr{notes} = w
   %    \alt  \alpha
 \\[2ex]
 \text{(Theory declarations)} &
 \mathit{tdec} & ::= & 
 \akey{theory}\ T\  \mathit{tattr}^+
\end{array}
\)
}

\begin{comment}
\[
\begin{array}{llll}
s \in \mathcal{S}, & \text{the set of sort symbols} &
n \in \mathcal{N}, & \text{the set of numerals} \\
u \in \mathcal{U}, & \text{the set of sort parameters \ \ } &
w \in \mathcal{W}, & \text{the set of Unicode character strings} \\
T \in \mathcal{T}, & \text{the set of theory names}
\end{array}
\]
\end{comment}

\newcommand{\logics}{
\(
\begin{array}{llcl}
 \text{(Logic attributes)} &
 \mathit{lattr} & ::= & \aattr{theories} = T^+
       \alt  \aattr{language} = w \\
   & & \alt& \aattr{extensions} = w
       \alt  \aattr{values} = w% \\
%   & & \alt& \aattr{notes} = w
%       \alt  \alpha
 \\[2ex]
 \text{(Logic declarations)} &
 \mathit{ldec} & ::= & \akey{logic}\ L\  \mathit{lattr}^+
\end{array}
\)
}

\begin{comment}
\[
\begin{array}{llll}
T \in \mathcal{T}, & \text{the set of theory names \ \ } &
w \in \mathcal{W}, & \text{the set of character strings} \\
L \in \mathcal{L}, & \text{the set of logic names} \\
\end{array}
\]
\end{comment}

\newcommand{\options}{
 \text{(Options)} &
  o & ::= &  \aattr{print-success} = b \\
  & & \alt&  \aattr{expand-definitions} = b \\
  & & \alt&  \aattr{interactive-mode} = b \\
% as removed:
%  & & \alt& \aattr{timeout} = r \\
%  & & \alt& \aattr{timeout} = \akey{none}\\
  & & \alt&  \aattr{produce-proofs} = b \\
  & & \alt&  \aattr{produce-unsat-cores} = b \\
  & & \alt&  \aattr{produce-models} = b \\
  & & \alt&  \aattr{produce-assignments} = b \\
  & & \alt& \aattr{regular-output-channel} = w \\
  & & \alt&  \aattr{diagnostic-output-channel} = w \\
  & & \alt& \aattr{random-seed} = n \\
  & & \alt&  \aattr{verbosity} = n\\
  & & \alt&  \alpha
}

\newcommand{\infoNames}{
 \text{(Info flags)} &
  i & ::= & \aattr{all-statistics} \\
  %ct eliminated these flags
  %& & \alt& \aattr{decisions} \\
  %& & \alt& \aattr{conflicts} \\
  %& & \alt& \aattr{restarts} \\
  %& & \alt& \aattr{notes} \\
  %ct eliminated these flags for now
  %& & \alt& \aattr{time} \\
  %& & \alt& \aattr{memory} \\
  & & \alt& \aattr{error-behavior} \\
  & & \alt& \aattr{name} \\
  & & \alt& \aattr{authors} \\
  & & \alt& \aattr{version} \\
  & & \alt& \aattr{status} \\
  & & \alt& \aattr{reason-unknown} \\
  & & \alt& a
}


\newcommand{\commandOptions}{
\(
\begin{array}{llcl}
 \options
 \\[2ex]
 \infoNames
\end{array}
\)
}

\newcommand{\commands}{
\(
\begin{array}{llcl}
 \text{(Commands)} &
  c & ::= & \akey{set-logic}\ L\\
  & & \alt& \akey{set-option}\ o \\
  & & \alt & \akey{set-info}\ \alpha \\
  & & \alt & \akey{declare-sort}\ s\ n \\
  & & \alt & \akey{define-sort}\ s\ u^*\ \tau \\
  & & \alt & \akey{declare-fun}\ f\ \sigma^*\ \sigma \\
  & & \alt & \akey{define-fun}\ f\ (x{:}\sigma)^*\ \sigma\ t \\
  & & \alt & \akey{push}\ n \\
  & & \alt & \akey{pop}\ n \\
  & & \alt & \akey{assert}\ t \\
  %ct & & \alt & \akey{assert}\ t\ l \\
  & & \alt & \akey{check-sat}  \\
  %ct & & \alt & \akey{keep-model}\\
  & & \alt & \akey{get-assertions} \\
  %ct  & & \alt & \akey{get-model}\ t^* \\     
  & & \alt & \akey{get-value}\ t^+ \\
  & & \alt & \akey{get-proof} \\
  %ct  & & \alt & \akey{get-rank}\ s \\
  & & \alt & \akey{get-unsat-core}\\
  & & \alt & \akey{get-info}\ i \\
  & & \alt & \akey{get-option}\ a \\
  & & \alt & \akey{exit}
 \\[2ex]
 \text{(Scripts)} &
  \mathit{scr} & ::= & c^*
 \end{array}
\)
}

\newcommand{\gresponses}{
 \text{(General response)} &
  \mathit{gr} & ::= & \akey{unsupported}
      \alt  \akey{success}
      \alt  \akey{error}\ w
}

\newcommand{\csresponses}{
 \text{(check-sat response)} &
 \mathit{csr} & ::= & \akey{sat} \alt \akey{unsat} \alt \akey{unknown} 
}

\newcommand{\iresponses}{
 \text{(get-info response)} &
 \mathit{gir}  & ::= & i^+ 
 \\[2ex]
 \text{(Info response)} &
  i & ::= 
%as removed:
%& \aattr{decisions} = n \\
%  & & \alt& \aattr{conflicts} = n\\
%  & & \alt& \aattr{restarts} = n\\
%  & & \alt& \aattr{time} = r\\
%  & & \alt& \aattr{memory} = r\\
%  & & \alt
& \aattr{error-behavior} = \mathit{eb}\\
  & & \alt& \aattr{name} = w\\
  & & \alt& \aattr{authors} = w\\
  & & \alt& \aattr{version} = w\\
  & & \alt& \aattr{reason-unknown} = \mathit{ru}\\
%  & & \alt& \aattr{notes} = w \\
  & & \alt& \alpha
 \\[2ex]
 \text{(Error behavior)} &
 \mathit{eb} & ::= & \akey{immediate-exit} \alt \akey{continue-execution}
 \\[2ex]
 \text{(Reason unknown)} &
 \mathit{ru} & ::= & %as removed: \akey{timeout}\alt
 \akey{memout} \alt \akey{incomplete}
}

\newcommand{\garesponses}{
 \text{(get-assertions response)} &
 \mathit{gar} & ::= & t^* 
}

\newcommand{\gpresponses}{
 \text{(get-proof response)} &
 \mathit{gpr} & ::= & p
}

\newcommand{\gucresponses}{
 \text{(get-unsat-core response)} &
 \mathit{gucr} & ::= & f^*
}

\newcommand{\gvresponses}{
 \text{(get-value responses)} &
 \mathit{gvr}  & ::= & (t\ t)^+
}

\newcommand{\gtaresponses}{
 \text{(get-assignment response)} &
 \mathit{gta} & ::= & (f\ b)^* 
}

\newcommand{\goresponses}{
 \text{(get-option response)} &
 \mathit{go} & ::= & v
}

\newcommand{\responses}{
\(
\begin{array}{llcl}
 \gresponses
 \\[2ex]
 \iresponses
 \\[2ex]
 \csresponses
 \\[2ex]
 \garesponses
 \\[2ex]
 \gpresponses
 \\[2ex]
 \gucresponses
 \\[2ex]
 \gvresponses
 \\[2ex]
 \gtaresponses
 \\[2ex]
 \goresponses
 \end{array}
\)
}

\begin{comment}
\[
\begin{array}{llll}
n \in \mathcal{N}, & \text{the set of numerals\ \ \ }  &
w \in \mathcal{W}, & \text{the set of character strings}
\end{array}
\]
\end{comment}




