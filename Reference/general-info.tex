% general-info.tex

%!TEX root = main.tex

%===============================================================================
\chapter{General Information}
%===============================================================================
\thispagestyle{empty}

%\rem{CT}{Chapter synopsis}

%-------------------------------------------------------------------------------
\section{About This Document}
%-------------------------------------------------------------------------------

This document is mostly self-contained, though 
it assumes some familiarity with first-order logic, \emph{aka} predicate calculus,
and higher-order logic, \emph{aka} simple type theory.
The reader is referred to any of several textbooks 
on first-order logic~\cite{Gal-86,Fit-96,Hen-01,Men-09} and 
monographs of higher-order logic \todo{add references}. 
Previous knowledge of Version 1.2 of the SMT-LIB standard~\cite{RanTin-RR-06}
is not necessary.
In fact, Version 1.2 users are warned that this version,
while largely based on Version 1.2, 
is \emph{not} backward compatible with it.
See the Version 2.0 document~\cite{BarST-RR-10} for a summary of the major differences.

This document provides BNF-style abstract and concrete syntax 
for a number of SMT-LIB languages.
\emph{Only the concrete syntax is part of the official SMT-LIB standard.}
The abstract syntax is used here mainly for descriptive convenience;
adherence to it is not prescribed.
Implementors are free to use whatever internal structure 
they please for their abstract syntax trees.
\medskip

New releases of the document are identified by their release date.
Each new release of the same version of the SMT-LIB standard 
contains, by and large, only \emph{conservative} additions and 
changes with respect to the standard described in the previous release,
as well as improvements to the presentation.
The only non-conservative changes may be error fixes.

Historical notes and 
explanations of the rationale of design decisions
in the definition of the SMT-LIB standard are provided 
in Appendix~\ref{chap:notes}, 
with reference in the main text given as a superscript number 
enclosed in parentheses.

%-------------------------------------------------------------------------------
\subsection{Change log for Version 2.7}
%-------------------------------------------------------------------------------

\paragraph{Release: \todo{Add release No.}}
\begin{itemize}
\item First release of Version 2.7.
\end{itemize}

%-------------------------------------------------------------------------------
\subsection{Differences between Version 2.7 and Version 2.6}
%-------------------------------------------------------------------------------

\todo{Differences.}

%-------------------------------------------------------------------------------
\subsection{Change log for Version 2.6}
%-------------------------------------------------------------------------------

\paragraph{Release: 2021-05-12}
\begin{itemize}
\item Fixed misleading example of usage of \expr{as} to disambiguate symbol sorts.
\end{itemize}

\paragraph{Release: 2021-04-02}
\begin{itemize}
\item Now allowing reserved words in s-expression (as arguments of attributes).
\item Disambiguated description of in-line definition (\attr{named} annotation).
\end{itemize}


\paragraph{Release: 2017-07-18}
\begin{itemize}
\item First release of Version 2.6.
\end{itemize}

%-------------------------------------------------------------------------------
\subsection{Differences between Version 2.6 and Version 2.5}
%-------------------------------------------------------------------------------

The SMT-LIB 2 language and logic now supports user-defined algebraic datatypes.
Such types can be introduced with two new commands:
\ter{declare-datatype}, to declare a single algebraic datatype, and
\ter{declare-datatypes}, to declare two or more mutually recursive datatypes
(see Section~\ref{sec:new-symbols}).
The language of terms now includes a new binder, \ter{match}, 
for pattern-matching-based case analysis over datatype values
(see Section~\ref{sec:binders}).
The variables bound by \ter{match} are those that occur in \emph{patterns},
also a new addition to the language.
The underlying logic has also been modified to provide built-in semantics
for the constructor, selector and tester symbols defined 
with each new algebraic datatype
(see Chapter~\ref{chap:logical-semantics}).

This version also makes official a number of \ter{set-info} attributes
used in benchmarks from the official SMT-LIB repository
and specifies some requirements on their occurrence and order
(see Section~\ref{sec:benchmarks} and Section~\ref{sec:script-info}).
Any other differences in this document are only edits to improve 
the presentation.
Except for the addition of algebraic datatypes, 
which is fully backward compatible,
the rest of the SMT-LIB language is unchanged.


%-------------------------------------------------------------------------------
\subsection{Change log for Version 2.5}
%-------------------------------------------------------------------------------

\paragraph{Release: 2015-06-28}
\begin{itemize}
\item 
Clarified in Section~\ref{sec:modes} that the \ter{exit} command can be issued in any mode.
\end{itemize}


\paragraph{Release: 2015-05-28}
\begin{itemize}
\item 
First release of Version 2.5.
\end{itemize}

%-------------------------------------------------------------------------------
\subsection{Differences between Version 2.5 and Version 2.0}
%-------------------------------------------------------------------------------

Version 2.5 is an extension of Version 2.0 and, with two minor exceptions,
is fully backward compatible with it.
There is then no need to have separate support for 2.0 
if one supports Version 2.6.
The following list summarizes notable differences and extensions.
The first two items are the only non-backward compatible changes.


\begin{itemize}
\item
There is now a different set of escape sequences for string literals.
It consists of a single sequence, \ter{""}, used to represent the double quote
character within the literal.

\item
The predefined option \attr{expand-definitions} has been removed
because there are now no cases in which it applies. 

\item
SMT-LIB source files are not limited to the US-ASCII format anymore
and can now consist of Unicode characters.  
The concrete encoding is currently left unspecified, but should be 
a compatible 8-bit extension of the 7-bit US-ASCII set, such as UTF-8.

\item
We have clarified several points about the character set used 
by the SMT-LIB language and specified more precisely which characters are allowed
in string literals, identifiers and symbols.

\item
We have made explicit several details on the scoping and shadowing rules
for identifiers, in particulars those occurring in binders.

\item
Identifiers can now be indexed not just with numerals but also with symbols.

\item
The use of the term attribute \attr{pattern} and its related syntax 
for quantifier patterns has been made official.

%\item
%The command \ter{set-info} has been renamed \ter{meta-info}.
%The old name is still accepted but its use is now deprecated.
 
\item
The solver option \attr{interactive-mode} has been renamed 
\attr{produce-assertions}.
The old name is still accepted but its use is now deprecated.

\item
There is now a predefined argument, \ter{ALL}, for the \ter{set-logic} command
which refers to the most general logic supported by the solver executing the command.

\item
We have introduced a notion of \emph{execution mode} for a solver
to better describe the restriction on when commands can be executed 
or options set.

\item
There is a new solver option \attr{global-declarations} 
that makes all definitions and declarations global and 
not removable by \ter{pop} operations.
Global declarations can be removed only 
by the new command \ter{reset}.

\item
The new command \ter{reset} brings the state of a solver
to the state it had immediately after start up (resetting everything).

\item
The new command \ter{reset-assertions} empties the assertion stack
and removes all assertions.
If \attr{global-declarations} is set to \ter{false}, it also removes 
all declarations and definitions.

\item
The new command \ter{check-sat-assuming} checks the satisfiability 
of the current context under an additional number of assumptions provided 
as input to the command.
When it returns \ter{unsat}, a new companion command, 
\ter{get-unsat-assumptions}, returns the subset of input assumptions
used by the solver to prove the context unsatisfiable. 
The latter command is enabled or disabled with the new option 
\attr{produce-unsat-assumptions}.
The old \ter{check-sat} command can now be defined, conservatively,
as a special case of \ter{check-sat-assuming} with an empty set of assumptions.

\item
The new command \ter{declare-const} can now be used to declare 
nullary function symbols.

\item
The new command \ter{echo} prints back on the regular output channel
a string provided as input.

\item
The new commands \ter{define-fun-rec} and \ter{define-funs-rec} respectively
allow the definition of recursive functions and of sets of mutually recursive
functions.

\item
The new command \ter{get-model} returns a representation of a model computed
by the solver in response to an invocation of the \ter{check-sat} or 
\ter{check-sat-assuming} command.

\item
The new \ter{get-info} flag \attr{assertion-stack-levels} returns 
the current number of levels in the assertion stack.

\item
The new option \attr{reproducible-resource-limit} can be used to set 
a solver-defined resource limit that applies to each invocation of 
\ter{check-sat} or \ter{check-sat-assumptions}.

\end{itemize}





%-------------------------------------------------------------------------------
\subsection{Typographical and notational conventions}
%-------------------------------------------------------------------------------

The concrete syntax of the SMT-LIB language is defined
by means of BNF-style production rules.
In the concrete syntax notation,
terminals are written in typewriter font, as in \ter{false},
while syntactic categories (non-terminals) are written 
in slanted font and enclosed in angular brackets, as in \nter{term}.
In the production rules, the meta-operators $::=$ and $\mid$ are used as usual 
in BNF. 
Also, as usual, the meta-operators $\_^*$ and $\_^+$ denote zero, respectively, 
one, or more repetitions of their argument.
We use $\_^n$ and $\_^{n+1}$ instead of $\_^*$ and $\_^+$ 
when we want to indicate that multiple occurrences to the latter operators 
have the same number of repetitions.
%We refer to input characters by their decimal or hexadecimal code 
%in the Extended ASCII character set.
We use the notation \dec{$d$} (resp., \hex{$e$}) to represent 
the Unicode character with decimal code $d$ (resp., hexadecimal code $e$).  Remember that the US-ASCII character with code $d < 128$ is also 
the \href{http://www.utf8-chartable.de/unicode-utf8-table.pl?utf8=dec}{Unicode character} \dec{$d$}.
Examples of concrete syntax expressions are provided in shaded boxes like the following.
\begin{lstlisting}[linewidth=10em]
(f (- x) x) 
\end{lstlisting}
 
In the abstract syntax notation,
which uses the same meta-operators as the concrete syntax,
words in $\akey{boldface}$ as well as the symbols $\eqs, \exists, \forall$, 
and $\Pi$ denote terminal symbols,
while words in $\mathit{italics}$ and Greek letters denote 
syntactic categories.
For instance, $x, \sigma$ are non-terminals and $\bool$ is a terminal.
Parentheses are meta-symbols, used just for grouping---they are not part
of the abstract language.
Function applications are denoted simply by juxtaposition, 
which is enough at the abstract level.

To simplify the notation, 
when there is no risk of confusion,
the name of an abstract syntactic category is also used,
possibly with subscripts, to denote individual elements of that category.
For instance, $t$ is the category of terms
and $t$ (as well as $t_1, t_2$ and so on) is also used to denote individual terms. 

The meta-syntax $\bar{x}$ denotes a sequence of the form $x_1 x_2 \cdots x_n$
for some $x_1, x_2, \ldots, x_n$ and $n \geq 0$.


%-------------------------------------------------------------------------------
\section{Overview of SMT-LIB}
%-------------------------------------------------------------------------------

Satisfiability Modulo Theories (SMT)\index{SMT} is an area of automated deduction 
that studies methods for checking the satisfiability of first-order formulas 
with respect to some logical theory $\T$ of interest~\cite{BarSST-09}.
What distinguishes SMT from general automated 
deduction is that the background theory $\T$ need not be finitely or even first-order 
axiomatizable, and that specialized inference methods are used for each theory. 
By being theory-specific and restricting their language to certain classes 
of formulas (such as, typically but not exclusively, quantifier-free formulas), 
these specialized methods can be implemented in solvers 
that are more efficient in practice than general-purpose theorem provers. 

While SMT techniques have been traditionally 
used to support deductive software verification, they have found applications 
in other areas of computer science such as, for instance, planning, model checking and 
automated test generation. 
Typical theories of interest in these applications
include formalizations of various forms of arithmetic, arrays, finite sets, 
bit vectors, algebraic datatypes, strings, floating point numbers,
equality with uninterpreted functions, and various combinations of these.


%-------------------------------------------------------------------------------
\subsection{What is SMT-LIB?}

%The main goal of the SMT-LIB\index{SMT-LIB} initiative~\cite{SMT},
%coordinated by these authors,
%is to provide an on-line library of benchmarks for 
%\emph{Satisfiability Modulo Theories}\index{satisfiability!modulo theories}.  

%A lot of work has been done in the last few years by several research groups
%on building systems for satisfiability modulo theories\cite{systems}.

%Examples of background theories typically used in computer science
%are the theory of real numbers, the theory of integer numbers,  
%and the theories of various data structures 
%such as lists, arrays, bit vectors and so on.

SMT-LIB is an international initiative,
coordinated by the authors of this document and 
endorsed by a large number of research groups world-wide,
aimed at facilitating research and development in SMT~\cite{SMT-LIB}\index{SMT-LIB}.
Since its inception in 2003, the initiative has pursued these aims
by focusing on the following concrete goals:
provide standard rigorous descriptions of background theories used in SMT systems;
develop and promote common input and output languages for SMT solvers;
establish and make available to the research community a large library of
benchmarks for SMT solvers. 

The main motivation of the SMT-LIB initiative was the expectation that 
the availability of common standards and of a library of benchmarks would 
greatly facilitate the evaluation and the comparison of SMT systems,
and advance the state of the art in the field,
in the same way as, for instance,
the TPTP library~\cite{Sut-JAR-09} has done for theorem proving,
or the SATLIB library~\cite{HooStu-SAT-00} did 
for propositional satisfiability.
These expectations have been largely met, 
thanks in no small part 
to extensive benchmark contributions from the research community and
to an annual SMT solver competition, SMT-COMP~\cite{BardMS-CAV-05},
based on benchmarks from the library.

At the time of this writing, 
the library contains more than 100,000 \todo{update} benchmarks and continues to grow.
Formulas in SMT-LIB format are accepted by the great majority 
of current SMT solvers.
Moreover, much published experimental work in SMT relies significantly 
on SMT-LIB benchmarks.


\subsection{Main features of the SMT-LIB standard}

The previous main version of the SMT-LIB standard, Version 1.2, 
provided a language for specifying theories, logics (see later), and benchmarks,
where a benchmark was, in essence, a logical formula to be checked for 
satisfiability with respect to some theory.

Version 2.0 sought to improve the usefulness of the SMT-LIB standard
by simplifying its logical language 
while increasing its expressiveness and flexibility.
In addition, 
it introduced a command language for SMT solvers that 
expanded their SMT-LIB interface considerably,
allowing users to tap the numerous functionalities 
that most modern SMT solvers provide.

Like Version 2.0 and later versions, Version 2.t defines:
\begin{itemize}
\item
a language for writing \emph{terms and formulas} in a sorted (i.e., typed) 
version of first-order logic;
\item
a language for specifying \emph{background theories} and 
fixing a standard vocabulary of sort, function, and predicate symbols for them;
\item
a language for specifying \emph{logics}, 
suitably restricted classes of formulas to be checked 
for satisfiability with respect to a specific background theory;
\item
a \emph{command} language for interacting with SMT solvers
via a textual interface
that allows asserting and retracting formulas, 
querying about their satisfiability, examining their models or 
their unsatisfiability proofs, and so on.

\end{itemize}

