%%%%%%%%%%%%%%%%%%%%%%%%%%%%%%%%%%%%%%%%%%%%%%%%%%%%%%%%%%%%%%%
%
% Welcome to Overleaf --- just edit your LaTeX on the left,
% and we'll compile it for you on the right. If you give
% someone the link to this page, they can edit at the same
% time. See the help menu above for more info. Enjoy!
%
% Note: you can export the pdf to see the result at full
% resolution.
%
%%%%%%%%%%%%%%%%%%%%%%%%%%%%%%%%%%%%%%%%%%%%%%%%%%%%%%%%%%%%%%%
% EPC flow charts
% Author: Fabian Schuh
\documentclass{minimal}

\usepackage{pgf}
\usepackage{tikz}
%%%<
\usepackage{verbatim}
\usepackage[active,tightpage]{preview}
\PreviewEnvironment{tikzpicture}
\setlength\PreviewBorder{5pt}%
%%%>

\begin{comment}
:Title:  EPC flow charts
:Grid: 1x2


\end{comment}
\usepackage[utf8]{inputenc}
\usetikzlibrary{arrows,automata}
\usetikzlibrary{positioning}


\tikzset{
    state/.style={
           rectangle,
           rounded corners,
           draw=black, very thick,
           minimum height=2em,
           inner sep=2pt,
           text centered,
           },
}

\begin{document}

\begin{tikzpicture}[->,>=stealth']

 % Position of QUERY 
 % Use previously defined 'state' as layout (see above)
 % use tabular for content to get columns/rows
 % parbox to limit width of the listing
 \node[state] (ASSERT) 
 {\begin{tabular}{l}
  \textbf{Assert mode}\\
   \parbox{4cm}{} %\begin{itemize}
%    \item Start
%    \item Parameter $Q$
%    \item Zufallszahl aus \mbox{$[0, 2^Q-1]$} in Slotz�hler $SC$
%   \end{itemize}
%   }\\[4em]
%   \textbf{QueryAdjust}\\
%   \parbox{4cm}{\begin{itemize}
%    \item Variiere Q
%    \item neue Zufallszahl
%   \end{itemize}
%  }
 \end{tabular}};
  
 % State: SAT with different content
 \node[state,    	% layout (defined above)
  text width=3cm, 	% max text width
  yshift=2cm, 		% move 2cm in y
  right of=ASSERT, 	% Position is to the right of ASSERT
  node distance=6.5cm, 	% distance to ASSERT
  anchor=center] (SAT) 	% posistion relative to the center of the 'box'
 {%
 \begin{tabular}{l} 	% content
  \textbf{Sat mode}\\
  \parbox{2.8cm}{} %{Best�tigen mit $RN_{16}$}
 \end{tabular}
 };
 
 % STATE UNSAT
 \node[state,
  below of=SAT,
  yshift=-3cm,
  anchor=center,
  text width=3cm] (UNSAT) 
 {%
 \begin{tabular}{l}
  \textbf{Unsat mode}\\
  \parbox{2.8cm}{} %{Dekrementiere Slotz�hler}
 \end{tabular}
 };

%  % STATE EPC
%  \node[state,
%   right of=ACK,
%   node distance=5cm,
%   anchor=center] (EPC) 
%  {%
%  \begin{tabular}{l}
%   Antwort: \textbf{EPC}\\
%   \parbox{4cm}{Mit n�chstem \mbox{\textbf{QueryRep}} als "`inventoried"' markieren.}
%  \end{tabular}
%  };

 % draw the paths and and print some Text below/above the graph
 \path 
 (ASSERT) 	edge[bend left=20]  node[anchor=south,above]{(1)}  (SAT)
 (ASSERT)   edge[bend right=20] node[anchor=south,above]{(7)} (UNSAT)
 (ASSERT)  	edge[loop below]    node[anchor=north,below]{(6)} (ASSERT)
 
 (SAT)   	edge[bend left=20]  node[anchor=south,above]{(2)}  (ASSERT)
 (SAT)  	edge[loop above]    node[anchor=sourth,above]{(3)} (SAT)

 (UNSAT)   	edge[bend right=20]  node[anchor=north,below]{(5)}  (ASSERT)
 (UNSAT)  	edge[loop below]    node[anchor=north,below]{(4)} (UNSAT)
 ;

\end{tikzpicture}



\end{document}