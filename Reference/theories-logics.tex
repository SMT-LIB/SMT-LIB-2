% theories-logics.tex

%!TEX root = reference-manual.tex

%===============================================================================
\chapter{Theory and Sublogic Specifications} 
%===============================================================================
\thispagestyle{empty}


%-------------------------------------------------------------------------------
\section{Theory Declarations}
%-------------------------------------------------------------------------------

\begin{figure}
\theories
\caption{Abstract syntax for theories}
\label{fig:abstract-theories}
\end{figure}


The catalog of SMT-LIB theories consists of theory declarations
in a format specified in this section.
In the previous version of the SMT-LIB standard,
a \define{theory declaration} defines both 
a many-sorted signature for a theory and the theory itself.
The signature is specified by individual declarations of sort 
and function symbols with their rank.
In this version, additionally,
a function symbol declaration can be \emph{schematic} and 
stand in fact for an infinite number of them.
This is achieved with the use of \emph{parametric ranks}, as explained below.
A parametric rank stands for a family of ranks
obtained by instantiating its parameters with all sorts in the signature.

The abstract syntax for theory declarations
is provided in Figure~\ref{fig:abstract-theories}.
The syntax defines an attribute-value-based format.
In addition to the symbols and syntactic categories defined previously,
the production rules in the figure assume as given 
the following sets of symbols:
\begin{itemize}
\item
the set $\mathcal{N}$ of all \define{numerals};
\item
an infinite set $\mathcal{W}$ of \define{character strings},
meant to contain free text;
\item
an infinite set $\mathcal{T}$ of \define{theory names},
used to give a name to each basic theory;
\item
an infinite set $\mathcal{U}$ of \define{sort parameters},
used in schematic function symbol declarations.
\end{itemize}

The terminals 
$\aattr{funs}$, 
$\aattr{axioms}$, 
$\aattr{notes}$, 
$\aattr{sorts}$,
and
$\aattr{definition}$
are reserved attribute names from $\mathcal{A}$. 
The sets of values for the attribute they name are as specified in the rules.
In addition to the predefined attributes,
a theory declaration $D_T$ can contain an unspecified number of 
user-defined attributes, and their values---formalized in the grammar 
simply as annotations $\alpha$.
Note that
in the abstract syntax of a theory declaration,
attribute-value pairs can appear in any order.
However, they are subject to the restrictions below.

\begin{definition}[Theory Declarations]
\label{def:theory}
The only legal theory declarations in the SMT-LIB format are those 
that satisfy the following restrictions.
\begin{enumerate}
\item
They contain at least one instance of the $\aattr{sorts}$ attribute and 
exactly one instance of the $\aattr{definition}$ attribute\footnote{
Which makes both attributes non-optional.
}.

\item
Each sort symbol used in a $\aattr{funs}$ attribute is declared 
in some $\aattr{sorts}$ attribute.

\item
The definition of the theory, however provided in the $\aattr{definition}$ attribute,
refers only to sort and function symbols formally declared
in $\aattr{sorts}$ and $\aattr{funs}$ attributes or 
mentioned in $\aattr{sorts-description}$ and $\aattr{funds-description}$ attributes.

\item
The formulas in the $\aattr{axioms}$ attributes, if any,
are built over the symbol declared in the $\aattr{sort}$ and $\aattr{funs}$
attributes.

\item
In parametric function symbol declarations $\Pi\, u^+\: f\: \tau^+\: \alpha^*$,
all sort parameters occurring in $\tau^+$ are listed in $u^+$. 
\end{enumerate}
\end{definition}


Some attributes, such as $\aattr{definition}$ for instance,
are \define{informal attributes} in the sense that 
their value ($w$) is free text.
When possible,
the signature of a theory is defined by 
the collection of all the $\aattr{sorts}$ and $\aattr{funs}$ 
in the theory declaration.
This is the case for theories with a finite signature.
However, some SMT-LIB theories have an infinite number of sort 
or function symbols.
In that case, 
those symbols are defined informally, in plain text,
in $\aattr{sorts-description}$ and $\aattr{funds-description}$ attributes.

The value of a $\aattr{sorts}$ attribute is a non-empty sequence
of sort symbol declarations $\mathit{sdec}$.
A declaration $\mathit{sdec}$ declares 
a sort symbol $s$ to have some arity $n$ and 
contains zero or more annotations $\alpha$.
In this version, there are no predefined annotations for sort declarations.

The value of a $\aattr{funs}$ attribute is a non-empty sequence
of possibly parametric function symbol declarations $\mathit{pdec}$.
A non-parametric function symbol declaration $\mathit{fdec}$ 
declares a function symbol $f$ to have a rank $\sigma^+$,
and contains zero or more annotations.
A parametric function symbol declaration $\Pi\, u^+\: (f\: \tau^+\: \alpha^*)$,
on the other hand,
declares a class of function symbols, 
all named $f$ and each with a rank obtained from $\tau^+$ 
by instantiating the parameters of $\tau^+$ with (non-parametric) sorts
in the theory.
For every $n > 0$ and $u_1, \ldots, u_n \in \mathit{U}$
the prefix construct $(\Pi\:u_1\:\cdots\:u_n\ \_)$ is 
a \define{universal sort binder (or quantifier)}\index{binder!sort binder}
for $u_1, \ldots, u_n$.

There are three predefined annotations for function symbol declarations.
All of them, however, concern the concrete syntax,
so they are discussed in Chapter~\ref{chap:concrete-syntax}.

The $\aattr{funs}$ attributes is optional in theory declaration
because a theory might lack function symbols---although 
such a theory would not be not very interesting.
The $\aattr{sorts}$ attribute, however, is not optional
because sorted frameworks require the existence of at least one sort.\note{at-least-one-sort}

The non-optional $\aattr{definition}$ attribute is meant to contain
a natural language definition of the theory.
While this definition is expected to be as rigorous as possible,
it does not have to be a formal one.
Some theories (such as the theory of integer numbers) are well known,
hence just a reference to their common name might be enough.
For theories that have a small set of axioms (or axiom schemas),
it might be convenient to list the actual axioms in that attribute.
For other theories, 
a mix of formal notation and informal explanation might be more appropriate.
  
Formal first-order axioms 
that define a theory, or part of it, can be additionally
provided in the optional $\aattr{axioms}$ attribute 
as a list of \emph{closed formulas},
that is, well-sorted terms of sort \ter{Bool} all of whose variables are bound.


The optional attribute $\aattr{notes}$ is meant 
to contain documentation information on the theory declaration 
such as authors, date, version, references, etc.,
although this information can also be provided 
with more specific user-defined attributes.


%-------------------------------------------------------------------------------
\subsection{The Core theory}
%-------------------------------------------------------------------------------

%\begin{figure}[t]
%\[
%\begin{array}{l}
%\akey{theory}\ \akey{Core} \\
%\ 
%\begin{array}{ll}
%\aattr{sorts}&  \bool\ 0 \\
%\akey{funs} & \top\ \bool \quad
%              \bot\ \bool \quad
%              \lnot\ \bool\ \bool \\
%            & \land\ \bool\ \bool\ \bool \quad
%              \lor\ \bool\ \bool\ \bool \\
%            & \lxor\ \bool\ \bool\ \bool \quad
%              \limplies\ \bool\ \bool\ \bool \\
%            & \Pi\ u\ (\akey{ite}\ \bool\ u\ u) \\
%\akey{definition} 
%& \text{$\bool$ is the two-element domain of Boolean values.} \\
%& \text{For any sort $\sigma$, $(\akey{ite}\ \bool\ \sigma\ \sigma)$ is the function that returns}\\
%& \text{its second argument or its third depending on whether}\\
%& \text{the first argument evaluates to true or not.} \\
%& \text{The other function symbols are defined as usual} \\
%& \text{(with $\lxor$ being exclusive or).}
%\end{array}
%\end{array}
%\]
%\caption{The $\akey{Core}$ theory}
%\label{fig:core-theory}
%\end{figure}


To provide the usual set of Boolean connectives for building formula,
this version defines a basic core theory 
which is implicitly included in every other SMT-LIB theory.
More precisely, every theory declaration is implicitly assumed to contain
the $\aattr{sorts}$ and $\aattr{funs}$ attributes of the $\akey{Core}$ theory
defined in Figure~\ref{fig:core-theory},
and to define those symbols in the same way as in $\akey{Core}$.

Note that $\akey{Core}$ does not define a symbol for double implication.
Such a connective is superfluous because now the equality symbol $\eqs$ 
can be used in its place. 
Similarly, 
the if-then-else connective of Version 1.2 is also absent
because $\akey{ite}$ can now be used with formulas as well.

%NOTE: The theory Empty currently in SMT-LIB is superseded by the theory declaration schema above.



%-------------------------------------------------------------------------------
\section{Sublogics} \label{sec:sublogics}  % referenced in commands.tex
%-------------------------------------------------------------------------------

\begin{figure}
\logics
\caption{Abstract syntax for logics}
\label{fig:abstract-logics}
\end{figure}

% as remove for now:
% \trem{Revise section}

%The SMT-LIB format allows the explicit definition of sublogics of 
%its main logic---many-sorted first-order logic with equality---that 
%restrict both the main logic's syntax and semantics.
%A new (sub)logic is defined in the SMT-LIB language by 
%a \emph{logic declaration}
%whose abstract syntax is provided in Figure~\ref{fig:abstract-logics}.

%In addition to the already defined symbols and syntactical categories,
%the production rules in Figure~\ref{fig:abstract-theories} assume as given 
%an infinite set $\mathcal{L}$ of \emph{logic names},
%used to give a name to each logic.
%In the rules,
%the letter $L$ denotes logic names.

%The symbols 
%$\aattr{theory}$, 
%$\aattr{language}$, 
%$\aattr{extensions}$, 
%and
%$\aattr{notes}$
%are reserved attribute symbols from $\mathcal{A}$. 
%Their sets of values are as specified in the rules.
%As for theories, 
%a logic declaration $D_L$ can contain an unspecified number of 
%user defined attributes, and their values---formalized in the grammar 
%simply as annotations $\alpha$.

%
%\begin{definition}[Logic Declarations]
%The only legal logic declarations in the SMT-LIB format are those 
%that satisfy the following restrictions:

%\begin{enumerate}

%\item
%They include a declaration of the attributes 
%$\aattr{theory}$ and $\aattr{language}$.

%\item 
%They contain at most one declaration per attribute.

%\item
%The value of the attribute $\aattr{theory}$ coincides with 
%the name of a theory $T$ for some theory declaration $D_T$ in SMT-LIB;
%\end{enumerate}
%\end{definition}

%The text attribute $\aattr{notes}$ serves the same purpose as in theory declarations. 

%The attribute $\aattr{theory}$ refers to a theory $T$ in SMT-LIB.
%As we will see,
%the effect of this attribute is to declare that the logic's
%sort, function and predicate symbols
%are restricted to those in the signature of $T$,
%and that
%the logic's semantics is restricted to the models of $T$.

%The attribute $\aattr{language}$ describes in free text
%the logic's \emph{language},
%that is,
%the specific subset of SMT-LIB formulas that are allowed in the logic.
%As explained in Section~\ref{sec:logics},
%this information is useful for tailoring solvers to the specific
%sublanguage of formulas used in a set of benchmarks.
%\extra{
%The attribute is text valued because it has mostly documentation purposes
%for the benefit of benchmark users.
%A natural language description of the logic's language seems therefore
%adequate for this purpose.
%However, we expect that the language will be at 
%least partially specified in some formal fashion in this attribute,
%for instance by using BNF rules.
%}

%It is understood that the formulas in the logic's language 
%are built over the signature $\Sigma$ of the associated theory
%unless this signature is explicitly expanded in the text of attribute $\aattr{language}$ with new\footnote{
%That is, \emph{uninterpreted}, or \emph{free}, in logic parlance.
%}
%symbols.
%In that case,
%\emph{the text must explicitly indicate which 
%new sort, function or predicate symbols
%are included in the expansion}.
%\extra{
%See later for why it is convenient to expand the language with 
%free symbols.
%}
%In effect then, an SMT-LIB logic's language is really parametric
%in such a signature expansion.

%
%The optional attribute $\aattr{extensions}$ is meant 
%to document any notational conventions, or syntactic sugar, allowed in
%formulas of this logic---and hence in benchmarks for the logic.
%\extra{
%This is useful because in common practice the syntax of 
%a logic is often extended for convenience with syntactic sugar.
%}

%One example of a logic declaration that uses the attributes
%$\aattr{theory}$, $\aattr{language}$ and $\aattr{extensions}$
%is what we could call \emph{linear Presburger arithmetic}.
%In this logic,
%the theory $T$ is the theory of natural numbers 
%with signature $\{0, s, +, <\}$ ($s$ for the successor function)
% as axiomatized by Presburger~\cite{pres}.
%The language is the set of Boolean combinations of atomic formulas over
%this signature expanded with an infinite number of free constants.
%Examples of syntactic conventions 
%defined in the $\aattr{extension}$ attribute could be 
%the use of a numeral $n$ for the $n$-fold application of $s$ to $0$,
%and the use of  the expression $n*t$ for the term 
%$\underbrace{t + \cdots + t}_{n \text{ times}}$.

%Note that 
%function or predicate symbols introduced as syntactic sugar are not formally
%defined in the $\aattr{funs}$ or $\aattr{pred}$ symbols, or anywhere else.
%In a sense, syntactic sugar does not officially exist, 
%even if it is allowed in benchmarks for convenience.
%\extra{
%This deficiency of the SMT-LIB language,
%which is meant to be resolved in future versions,
%is mostly due to the impossibility of 
%(formally) defining an infinite signature at the moment.
%That is why, for instance,
%in the example above numerals are introduced as syntactic sugar
%for ground terms over $\{s, 0\}$.
%}


