% underlying-logic.tex

%!TEX root = main.tex



%===============================================================================
\chapter{Logical Semantics of SMT-LIB Formulas} \label{chap:logical-semantics}
%===============================================================================
\thispagestyle{empty}

The underlying logic of the SMT-LIB language is a variant of many-sorted 
first-order logic (FOL) with equality~\cite{Man-MSL-93,Gal-86,Hen-01},
although it incorporates some syntactic \new{and semantic} features 
of higher-order logics:
in particular, the identification of formulas with terms 
of a distinguished Boolean sort, and the use of sort symbols of arity 
\begin{newver}
greater than 0,
as well as the use of functions as first-class values.
The latter is achieved by the addition of a background theory 
of higher-order functions,
which are modeled as values of sort \expr{(-> $\tau_1$ $\tau_2$)}.
\end{newver}


These features make for a more flexible and syntactically more uniform 
logical language.
However, while not exactly syntactic sugar, they do not change the essence 
of SMT-LIB logic with respect to traditional many-sorted FOL.
Quantifiers are still first-order, the sort structure is flat (no subsorts),
the logic's type system has no function types\footnote{\new{%
Note that arrow types, of the form \expr{(-> $\tau_1$ $\tau_2$)},
are defined in a theory, not in the base logic.
}}, 
no type quantifiers, no dependent types,
no provisions for parametric or subsort polymorphism.
The only polymorphism is of the \emph{ad-hoc} variety (a function symbol 
can be given more than one rank), although there is a syntactical mechanism 
for approximating parametric polymorphism.
As a consequence, all the classical meta-theoretic results from many-sorted 
FOL apply to SMT-LIB logic when considered in its full generality, that is,
with no restrictions on the possible models other than those imposed 
by the \akey{Core} \new{and the \akey{HO-Core} background theories introduced 
in Section~\ref{sec:core} and~\ref{sec:ho-core}}.
Those results still hold with \define{recursively axiomatizable} 
background theories, i.e., theories defined as the set of all models
of a recursive set of closed formulas (or \define{axioms}).
% As pointed out in Section~\ref{sec:theory-examples} the ability 
% to use non-recursively axiomatizable theories as background theories
% actually gives SMT-LIB logic the expressive power of higher-order logics.
% However, a formal treatment of this aspect is beyond the scope of this document.
\medskip

To define SMT-LIB logic and its semantics it is convenient to work 
with a more abstract syntax than the concrete S-expression-based syntax 
of the SMT-LIB language.
The formal semantics of concrete SMT-LIB expressions is then given by means 
of a translation into this abstract syntax.
A formal definition of this translation might be provided in later releases 
of this document.
Until then, we will appeal to the reader's intuition and on the fact that 
the translation is defined as one would expect.

The translation also maps concrete predefined symbols and keywords
to their abstract counterpart.
To facilitate reading, usually the abstract version of 
a predefined concrete symbol is denoted by the symbol's name 
in Roman bold font (e.g., \akey{Bool} for \ter{Bool}).
The same is done for keywords (e.g., \akey{definition} for \attr{definition}).

To define our target abstract syntax we start by fixing the following pairwise disjoint
sets of (abstract) symbols and values:
\begin{itemize}
\item
an infinite set $\so$ of \define{sort symbols} $s$ 
containing the symbol $\bool$,

\item
an infinite set $\sop$ of \define{sort parameters} $u$,

\item
an infinite set $\va$ of \define{variables}\index{variable} $x$,

\item
an infinite set $\fu$ of \define{function symbols}\index{symbol!function symbol} $f$
containing the symbols $\eqs$, $\land$, and $\lnot$,

\item
an infinite set $\mathcal{A}$ of \define{attribute names}\index{attribute name} $a$, 

\item
an infinite set $\val$ of \define{attribute values}\index{attribute values} $v$,

\item
the set $\st$ of \define{Unicode character strings} $w$,

\item
a two-element set $\bo = \{\true, \false\}$ of \define{Boolean values} $b$,

\item
the set $\na$ of \define{natural numbers} $n$,

\item
an infinite set $\te$ of \define{theory names} $T$,

\item
an infinite set $\lo$ of \define{logic names} $L$.


\end{itemize}


%-------------------------------------------------------------------------------
\section{The language of sorts}
%-------------------------------------------------------------------------------

\begin{figure}[t]
\sortterms
\caption{Abstract syntax for sort terms}
\label{fig:abstract-sorts}
\end{figure}

In many-sorted logics, terms are typed, or \define{sorted}. 
Each sort, which stands for a non-empty set of elements, is denoted 
by a sort symbol.
In SMT-LIB logic, the language of sorts is extended from sort symbols 
to \emph{sort terms} built with symbols from the set $\so$ above.
Formally, we have the following.

\begin{definition}[Sorts]
For all non-empty subsets $S$ of $\so$ and 
all mappings $\ar:S \to \mathbb{N}$,
the set $\sortt{S, \sop}$ of all \define{sorts}\index{sort} over $S$ and $\sop$ 
(with respect to $\ar$)
is defined inductively as follows:
\begin{enumerate}
\item 
every $u \in \sop$ is a sort;
\item
every $s \in S$ with $\ar(s) = 0$ is a sort;
\item
If $s \in S$ and $\ar(s) = n > 0$ and
$\tau_1, \ldots, \tau_n$ are sorts,
then the term
$s\,\tau_1\,\cdots\,\tau_n\:$ is a sort.
\end{enumerate}
We say that $s \in S$ has (or is of) \define{arity} $n$ 
if $\ar(s) = n$.
A sort is \emph{polymorphic} 
if it contains a sort parameter, it is \emph{monomorphic} otherwise.
\qed
\end{definition}

As an example of a sort,
if $\akey{Int}$ and $\akey{Real}$ are sort symbols of arity 0, 
$\akey{List}$ is a sort symbols of arity 1, and
$\akey{->}$ and $\akey{Array}$ are sort symbols of arity 2,
then the expression
\[
  (\akey{->}\ \akey{Int}\ \akey{List}\ (\akey{Array}\ \akey{Int}\ (\akey{List}\ \akey{Real})))
\] 
and all of its subexpressions are sorts.

Note that function symbol declarations in theory declarations (defined later)
use also polymorphic sorts.
Similarly to the example above,
if $u_1, u_2$ are parameters, that is elements of $\sop$,
the expression
\[
  (\akey{->}\ u_1\ u_2\ \akey{List}\ (\akey{Array}\ u_1\ (\akey{List}\ u_2)))
\]
and all of its subexpressions are polymorphic sorts.

An abstract syntax for monomorphic sorts $\sigma$ and polymorphic sorts $\tau$,
which ignores arity constraints for simplicity, 
is provided in Figure~\ref{fig:abstract-sorts}.
Note that every monomorphic sort is a polymorphic sort, but not vice versa.
In the following, we just use ``sort'' to refer to possibly polymorphic sorts
and we use ``monorphic sort'' for the more restricted case as needed.

%
%\trem{More}

%\subsection{Concrete to abstract syntax mapping}


%-------------------------------------------------------------------------------
\section{The language of terms} \label{sec:language}
%-------------------------------------------------------------------------------

\begin{figure}
\terms
\caption{Abstract syntax for unsorted terms}
\label{fig:abstract-terms}
\end{figure}

%\emph{Terms} are the main syntactical category of the SMT-LIB logic language.
In the abstract syntax,
terms are built out of variables from $\va$, 
function symbols from $\fu$, 
%a distinguished binary symbol ($\eqs$) for equality,\index{symbol!equality symbol}
and a set of \emph{binders}.
%
The logic considers, in fact, only \emph{well-sorted (polymorphic) terms},
a subset of all possible terms determined 
by a \emph{sorted signature}, as described below.

The set of all terms is defined by the abstract syntax rules of 
Figure~\ref{fig:abstract-terms}.
The rules do not distinguish between constant and function symbols
(they are all members of the set $\fu$). 
These distinctions are really a matter of arity, 
which is taken care of later by the well-sortedness rules.

\subsubsection{Binders}

For all $n \geq 0$,
distinct variables $x, x_1, \ldots, x_n \in \va$ and sort $\tau$,


\begin{itemize}
\item
the prefix construct
\ $\lambda\,x{:}\tau\ \_$ \ is 
a \define{function abstraction binder}\index{binder!lambda binder}
for $x$;

\item
the prefix construct
\ $\exists\,x{:}\tau\ \_$ \ is 
a \define{sorted existential binder (or existential quantifier)}\index{binder!existential binder}
for $x$;

\item
the prefix construct
\ $\forall\,x{:}\tau\ \_$ \ is 
a \define{sorted universal binder (or universal quantifier)}\index{binder!universal binder}
for $x$;

\item
the mixfix construct
\ $\akey{let}\,x_1 = \_\:\cdots\:x_n = \_\ \akey{in}\ \_$ \ is 
a \define{(parallel-)let binder}\index{binder!let binder}
for $x_1, \ldots, x_n$.

\item
the mixfix construct 
\ $\akey{match}\,\_\,\akey{with}\,p_1 \to \_\:\cdots\:p_n \to \_$ \ is 
a \define{match binder} for the variables that occur in the pattern $p_i$ for each $i=1,\ldots,n$;

\end{itemize}

Occurrences of variables in terms are defined to be \define{free}\index{free} or
\define{bound}\index{bound} as in the case of the concrete syntax;
the scope of each bound variable is defined similarly as well
(see Subsection~\ref{sec:free}).
Terms are \define{closed}\index{closed term} 
if they contain no free variables, and \define{open}\index{open term} otherwise.
Terms are \define{ground}\index{ground term} if they are variable-free.

For simplicity,
the defined language does not contain any logical symbols
other than the binders.
Logical connectives for negation, conjunction and so on 
and the equality symbol, which we denote here by $\eqs$,
are just function symbols of the basic theory \akey{Core},
implicitly included in all SMT-LIB theories
(see Subsection~\ref{sec:theory-examples}). 

\subsubsection{Annotations}

Terms can be optionally annotated with zero or more \define{attributes}\index{attribute}.
Attributes have no logical meaning, 
but they are a convenient mechanism for adding meta-logical information,
as illustrated in Section~\ref{sec:concrete-terms}. 
Syntactically, 
an attribute is either an attribute name $a \in \at$ or 
a pair the form $a = v$ where 
$a \in \at$ and 
$v$ is an attribute value in $\val$.\footnote{%
At this abstract level,
the syntax of attribute values is intentionally left unspecified.
}

Function symbols themselves may be annotated with a sort, as in $f^\tau$.
Sort annotations simplify the sorting rules of the logic,
which determine the set of well-sorted terms.


\subsection{Signatures} \label{sec:signature}

Well-sorted terms in SMT-LIB logic are terms that can be associated
with a unique sort by means of a set of \emph{sorting rules}
similar to typing rules in programming languages.
The rules are based on the following definition of 
a (many-sorted) signature.

\begin{definition}[SMT-LIB Signature]
\label{def:signature}
An \define{SMT-LIB signature}, or simply a \define{signature},
is a tuple $\Sigma$ consisting of:

\begin{itemize}
\item
a set $\sorts{\Sigma} \subseteq \so$ of \define{sort} symbols containing $\bool$;

\item
a set $\funs{\Sigma} \subseteq \fu$ of \define{function} symbols;
%%CT eliminated because unnecessary
%% these symbols are declared in the theory Core
% containing $\eqs$, $\land$, and $\lnot$;

\item
a distinguished finite set $\cons{\Sigma} \subseteq \funs{\Sigma}$ 
of \define{constructor} symbols;

\item
a distinguished finite set $\sels{\Sigma} \subseteq \funs{\Sigma}$ 
of \define{selector} symbols,
disjoint with $\cons{\Sigma}$;

\item
a distinguished finite set $\tests{\Sigma} \subseteq \funs{\Sigma}$ 
of \define{tester} symbols,
disjoint with $\cons{\Sigma}$ and $\sels{\Sigma}$, and with the same cardinality
as $\cons{\Sigma}$;

\item
a total mapping $\mathrm{con}_\Sigma:\sorts{\Sigma} \to 2^{\cons \Sigma}$,
assigning a (possibly empty) set of constructors to each sort symbol;

\item
a total mapping $\ar:\sorts{\Sigma} \to \mathbb{N}$,
assigning an arity to each sort symbol,
with $\ar(\bool) = 0$;

\item
a total mapping $\mathrm{sel}_\Sigma:\cons{\Sigma} \to (\sels \Sigma)^*$,
assigning a sequence of $n$ distinct selectors to each constructor of arity $n$
so that no selector is assigned to more than one constructor;

\item
a bijective mapping $\mathrm{tes}_\Sigma:\cons{\Sigma} \to \tests{\Sigma}$,
assigning a tester to each constructor;

\item
a partial mapping from $\va$ to $\sortt{\sorts\Sigma, \sop}$,
assigning a sort to some of the variables in $\va$;%
%so that there are infinitely many variables of sort $\tau$ 
%for each $\tau \in \sortt{\sorts\Sigma}$;
\footnote{%
Note that $\sortt{\sorts \Sigma}$, the set of all sorts over $\sorts\Sigma$, is 
non-empty because at least one sort in $\sorts{\Sigma}$, $\bool$, has arity 0.
}

\item
a left-total \define{ranking} relation\footnote{%
A binary relation $R \subseteq X \times Y$ is \define{left-total}
if for each $x \in X$ there is (at least) a $y \in Y$ such that $xRy$.
}
$R$ from $\funs{\Sigma}$ to $\sortt{\sorts \Sigma, \sop}^+$,
assigning at least one \define{rank} to each function symbol and
such that
\begin{enumerate}
%%CT eliminated because unnecessary
%\item
%$({\lnot},\, \bool\,\bool),\; ({\land},\, \bool\,\bool\,\bool) \in R$;
%
%\item
%$(\eqs,\, \tau\tau\,\bool) \in R$ for all $\tau \in \sortt{\sorts \Sigma}$;

\item
for all constructors $c \in \cons{\Sigma}$, 
selectors $g_1 \cdots g_n = \mathrm{sel}_\Sigma(c)$, and
sorts $\tau_1,\ldots,\tau_n,\tau \in \sortt{\sorts \Sigma, \sop}$,
if $(c, \tau_1\cdots\tau_n\tau) \in R$
then $(g_i,\tau\,\tau_i) \in R$ for all $i=1,\ldots,n$;

\item
for all constructors $c \in \cons{\Sigma}$ and 
testers $p = \mathrm{tes}_\Sigma(c)$,
if $(c, \tau_1\cdots\tau_n\tau) \in R$
then $(p,\tau\,\bool) \in R$;

\item
there is no constructor $c \in \cons{\Sigma}$ such that
$(c, \bar{\tau}_1\tau), (c, \bar{\tau}_2\tau) \in R$ 
for distinct $\bar{\tau}_1$ and $\bar{\tau}_2$.\endnote{
Because of this constraint, the return sort of a constructor uniquely determines 
the sort of its arguments.
That removes the need to specify the sort of the pattern variables
in a \akey{match} expression.
}
\end{enumerate}
\end{itemize}

A sort symbol in $\Sigma$ is an \define{algebraic data type}
if it is assigned a non-empty set of constructors.
%\ctrem{notes on constructors, selectors and testers}
\qed
\end{definition}

\begin{remark}
The restrictions imposed on theory declarations and on the various commands
for declaring new symbols in SMT-LIB scripts make sure that 
the signature defined by an SMT-LIB script is in fact a signature 
in the sense of Definition~\ref{def:signature}.
\end{remark}


In the following, we will write $\sortt{\Sigma, \sop}$ as an abbreviation 
of $\sortt{\sorts \Sigma, \sop}$, and $\sortt{\Sigma}$ as the subset of all 
monorphic sorts in $\sortt{\Sigma, \sop}$.
We will work with \emph{ranked} function symbols and \emph{sorted} variables
in a signature.
Formally,
given a signature $\Sigma$,
a \define{ranked function symbol} is 
a pair $(f, \tau_1\cdots\tau_n\tau)$ in $\fu \times \sortt{\Sigma, \sop}^+$,
which we write as $f{:}\tau_1\cdots\tau_n\tau$.
A \define{sorted variable} is 
a pair $(x,\tau)$ in $\va \times \sortt{\Sigma, \sop}$,
which we write as $x{:}\tau$.
We write 
$f{:}\tau_1\cdots\tau_n\tau \in \Sigma$ and
$x{:}\tau \in \Sigma$
to denote that $f$ has rank $\tau_1\cdots\tau_n\tau$ in $\Sigma$ and 
$x$ has sort $\tau$ in $\Sigma$.

We will also consider signatures that differ from a given signature $\Sigma$
only by the sort they assign to variables,
as well as signatures that conservatively expand a given signature $\Sigma$
with additional sort and function symbols or additional ranks 
for $\Sigma$'s function symbols.

\begin{definition}[Signature variants and expansions] 
A signature $\Sigma'$ is a \define{variant} of a signature $\Sigma$ 
if it is identical to $\Sigma$ possibly except for its mapping 
from variables to sorts.

A signature $\Omega$ is a \define{expansion} of a signature $\Sigma$
if all of the following hold:
$\sorts{\Sigma} \subseteq \sorts{\Omega}$;
$\funs{\Sigma} \subseteq \funs{\Omega}$;
the sort symbols of $\Sigma$ have the same arity in $\Sigma$ and in $\Omega$;
every sort of $\Sigma$ has the same constructors in $\Omega$ that it has in $\Sigma$;
every constructor of $\Sigma$ has the same selectors and testers in $\Omega$ 
that it has in $\Sigma$;
for all $x \in \va$ and $\tau \in \sortt{\Sigma, \sop}$,
$x{:}\tau \in \Sigma$ iff $x{:}\tau \in \Omega$;
for all $f \in \fu$ and $\bar{\tau} \in \sortt{\Sigma, \sop}^+$,
if $f{:}\bar{\tau} \in \Sigma$ then $f{:}\bar{\tau} \in \Omega$.
In that case, $\Sigma$ is a \define{subsignature} of $\Omega$.
\qed
\end{definition}


\subsubsection{Overloading}
The rank of a function symbol in a signature specifies, in order,
the expected sort of the symbol's arguments and result.
Note that it is possible for a function symbol
to be \define{overloaded}\index{overloaded function symbol} in a signature $\Sigma$
by being associated to more than one rank in $\Sigma$.
This form of \emph{ad-hoc polymorphism} is entirely unrestricted: 
a function symbol can have completely different ranks---even varying in arity. 
For example,
in a signature with sorts $\akey{Int}$ and $\akey{Real}$ (with the expected meaning),
it is possible for the minus symbol $-$ to have all of the following ranks:  
$\akey{Real}\:\akey{Real}$ (for unary negation over the reals),
$\akey{Int}\:\akey{Int}$ (for unary negation over the integers),
$\akey{Real}\:\akey{Real}\:\akey{Real}$ (for binary subtraction over the reals),
and
$\akey{Int}\:\akey{Int}\:\akey{Int}$ (for binary subtraction over the integers).

% Together with the mechanisms used 
% to declare theories (described in Section~\ref{sec:theories})
% and algebraic datatypes (described in Section~\ref{sec:new-symbols}),
% overloading also provides an approximate form of \emph{parametric polymorphism}
% by allowing one to declare function symbols with ranks
% all having the \emph{same shape}.
% For instance,
% it is possible to declare an array access symbol with rank
% $(\akey{Array}\:\tau_1\:\tau_2)\:\tau_1\:\tau_2$ 
% for all sorts $\tau_1, \tau_2$ in a theory signature. 
% %
% Strictly speaking, this is still ad-hoc polymorphism
% because SMT-LIB logic itself does not allow parametric sorts.\footnote{%
% Parametric sort terms that occur in theory declarations 
% and algebraic datatype declarations
% are meta-level syntax as far as SMT-LIB logic is concerned.
% They are \emph{schemas} standing for concrete sorts.
% }
% However, it provides most of the convenience of parametric polymorphism 
% while remaining within the confines of the standard semantics
% of many-sorted FOL.
% %\endnote{
% %\label{parametric-polymorphism}
% %\trem{Comment on why we do not have have real parametric polymorphism in SMT-LIB logic.}
% %}

A function symbol can be \define{ambiguous}\index{ambiguous symbol}
in an SMT-LIB signature for having distinct ranks of the form
$\bar{\tau}\tau_1$ and $\bar{\tau}\tau_2$ 
where $\bar{\tau}$ is a sequence of sorts and 
$\tau_1$ and $\tau_2$ are different sorts.
Thanks to the requirement in Definition~\ref{def:signature} that
variables have exactly one sort in a signature,
in signatures with no ambiguous function symbols
every term can have at most one sort.
In contrast,
with an ambiguous symbol $f$ whose different ranks are
$\bar{\tau}\tau_1, \ldots, \bar{\tau}\tau_n$,
a term of the form $f\, \bar{t}$,
where the terms $\bar t$ have sorts $\bar \tau$, can be given 
a unique sort only if $f$ is annotated with one of the result sorts
$\tau_1, \ldots, \tau_n$,
that is, only if it is written as $f^{\tau_i}\bar{t}$ 
for some $i \in \{i,\ldots, n\}$.
As a consequence, from now on we will assume 
that all ambiguous symbols are annotated as described above.

\subsection{Well-sorted terms}
 
\begin{figure}
 \termrules
\caption{Well-sortedness rules for terms.}
\label{fig:well-sorted-terms}
\end{figure}

Figure~\ref{fig:well-sorted-terms} provides a set of rules defining 
well-sorted terms with respect to an SMT-LIB signature $\Sigma$.
Strictly speaking then, and similarly to more conventional logics,
the SMT-LIB logic language is a family of languages parametrized 
by the signature $\Sigma$.
As explained later, 
for each script working in the context of a background theory $\T$,
the specific signature is jointly defined by the declaration of $\T$ plus
any additional sort and function symbol declarations contained in the script.

The format and meaning of the sorting rules in Figure~\ref{fig:well-sorted-terms} 
is fairly standard and should be largely self-explanatory 
to readers familiar with type systems.
In more detail, the letter $\tau$ (possibly primed or with subscripts) 
denotes sorts in $\sortt{\Sigma, \sop}$,
the letter $k$ denotes a natural number.
%
%the notation $x : \tau \in \Sigma$ means that 
%$\Sigma$ maps variable $x$ to sort $\tau$.
%The notation $f:\tau_1 \cdots \tau_{k+1} \in \Sigma$ means that 
%$f$ is an element of $\funs{\Sigma} \cup \{\eqs\}$ 
%and 
%has rank $\tau_1 \cdots \tau_{k+1}$ in $\Sigma$. 
Expressions of the form $\Sigma[x_1:\tau_1,\: \ldots,\: x_n:\tau_n]$ 
denotes the signature 
that maps $x_i$ to sort $\tau_i$ for $i=1,\ldots,n$, and 
coincides otherwise with $\Sigma$.
Finally, $\alpha^*$ denotes a possibly empty sequence of attributes.
The rules operate over \define{sorting judgments}
which are triples of the form $\Sigma \vdash t : \tau$.

\begin{definition}[Well-sorted Terms]
For every SMT-LIB signature $\Sigma$,
a term $t$ generated by the grammar in Figure~\ref{fig:abstract-terms}
is \define{well-sorted (with respect to $\Sigma$)}
if $\Sigma \vdash t : \tau$ is derivable 
by the sorting rules in Figure~\ref{fig:well-sorted-terms}
for some sort $\tau \in \sortt{\Sigma,\sop}$.
In that case, 
we say that \define{$t$ has, or is of, sort $\tau$}.
\qed
\end{definition}

With this definition, it is possible to show that every term has at
most one sort.\endnote{
  It would have been reasonable to adopt an
  alternative version of the rule for well-sortedness of terms
  $(f^\tau\; t_1\; \cdots\; t_k)\ \alpha^*$ with annotated function
  symbols $f^\tau$, without the second conjunct of the rule's side
  condition.  This would allow formation of terms with annotated
  function symbols $f^\tau$, even when $f$ lacked two ranks of the
  forms $\tau_1 \cdots \tau_k\tau$ and $\tau_1 \cdots
  \tau_k\tau'$, for distinct $\tau$ and $\tau'$.  The
  rationale for keeping this second conjunct is that with it, function
  symbols are annotated when used iff they are overloaded in this way.
  This means that it is clear from the use of the function symbol,
  whether or not the annotation is required.  This in turn should help
  to improve human comprehension of scripts written using overloaded
  function symbols. 
}

\begin{remark}[Match rules]
The two rules for the \akey{match} binder in Figure~\ref{fig:well-sorted-terms} 
require that the match cases be \define{exhaustive}:
every constructor term of sort $\delta$ must match one of the patterns;
but allow it to be \define{redundant}:
the same term may match more than one pattern.
Exhaustiveness is necessary to make sure each \akey{match} expression is
semantically well defined.
The first rule deals with \akey{match} expressions where no patterns consist 
of a variable.
In that case, exhaustiveness is enforced by requiring that each constructor
of the datatype appear in one of the patterns.
The second rule deals with \akey{match} expressions where one or more patterns 
consist of a variable.
In that case, exhaustiveness is guaranteed simply by the presence 
of those variables.
In both cases, the preconditions ensure that $\delta$ is a datatype, 
not just any sort, by requiring it to have a non-empty set of constructors.
\end{remark}


\begin{definition}[SMT-LIB formulas]
For each signature $\Sigma$,
the language of SMT-LIB logic is the set of all well-sorted terms
wrt $\Sigma$.
\define{Formulas} are well-sorted terms of sort $\bool$.
\qed
\end{definition}

In the following, we will use $\varphi$ and $\psi$ to denote formulas.
\medskip

\begin{constraint}
SMT-LIB scripts consider only closed formulas,
or \define{sentences}\index{sentence}, closed terms of sort $\bool$.\endnote{
\label{only-closed-formulas}
This is mostly a technical restriction,
motivated by considerations of convenience.
In fact,
with a closed formula $\varphi$ of signature $\Sigma$
the signature's mapping of variables to sorts is irrelevant.
The reason is that 
the formula itself contains its own sort declaration for its term variables,
either explicitly, for the variables bound by a quantifier, 
or implicitly, for the variables bound by a $\akey{let}$ binder.
Using only closed formulas then simplifies 
the task of specifying their signature,
as it becomes unnecessary to specify 
how the signature maps the elements of $\va$ to the signature's sorts.
}
\qed
\end{constraint}

There is no loss of generality in the restriction above because,
as far as satisfiability is concerned,
every formula $\varphi$ with free variables 
$x_1, \ldots, x_n$ of respective sort $\tau_1, \ldots, \tau_n$,
can be rewritten as 
\[
 \exists\: x_1{:}\tau_1 ( \ldots (\exists\:x_n{:}\tau_n\: \varphi\ )\ldots).
\]
An alternative way to avoid free variables in scripts is 
to replace them by fresh constant symbols of the same sort.
%defined within the script itself (see Section~\ref{sec:benchmarks}).
This is again with no loss of generality because,
for satisfiability modulo theories purposes,
a formula's free variables can be treated equivalently as 
\emph{free symbols} (see later for a definition).


%-------------------------------------------------------------------------------
\section{Structures and Satisfiability} \label{sec:models}
%-------------------------------------------------------------------------------

The semantics of SMT-LIB is essentially the same as that 
of conventional many-sorted logic,
relying on a similar notion of \emph{$\Sigma$-structure}.

%Given a polymorphic sort $\tau$, that is, a sort containing sort parameters in $\sop$,
%we consider the set $\sorttt{_\tau}{\Sigma}$ of all instances of this sort in $\sortt{\Sigma}$, that is,
%the set of all monomorphic sorts obtained by a complete substitution of sort parameters by sorts in 
%$\sortt{\Sigma}$.

A \define{sort substitution $\theta$ over a set $V$} of sort parameters in $\sop$
maps each sort parameter $u$ in $V$ to a sort $\theta(u)$ in $\sortt{\Sigma}$.
A sort instance of a polymorphic sort $\tau$ is a sort $\theta(\tau)$ obtained
by sort substitution, i.e.,
substituting every occurrence of all sort parameters $u$ by the associated sort
$\theta(u)$ in $\sortt{\Sigma}$.
%Given a polymorphic sort $\tau$, we define the set $\sorttt{_\tau}{\Sigma}$ of all instances of
%$\tau$ in $\sortt{\Sigma}$.
%TODO, same thing for functions

\begin{definition}[$\Sigma$-structure]
%\label{def:structure}
Let $\Sigma$ be a signature.
A \define{$\Sigma$-structure} $\str A$ is a pair consisting of 
a set $A$,  the \define{universe} of $\str A$, and 
a mapping that

\begin{itemize}
\item interprets
each $\sigma \in \sortt{\Sigma}$ as a \emph{non-empty} subset $\sigma^{\str A}$ of $A$,
which we call the \define{domain} of $\sigma$ in $\str A$,
with $A = \bigcup_{\sigma \in \sortt{\Sigma}} \sigma^{\str A}$;\endnote{
Distinct sorts can have non-disjoint domain in a structure.
However, whether they do that or not is irrelevant in SMT-LIB logic.
The reason is that 
the logic has no sort predicates, such as a subsort predicate, and 
does not allow one to equate terms of different sorts
(the term $t_1 \eqs t_2$ is ill-sorted unless $t_1$ and $t_2$ have 
the same sort).
As a consequence, a formula is satisfiable in a structure 
where two given sorts have non-disjoint domain 
iff it is satisfiable in a structure 
where the two sorts do have disjoint domains.
}

\item associates
an element $(f{:}\theta(\tau))^{\str A}$ of the set $\theta(\tau)^{\str A}$ to
each (ranked function symbol) $f{:}\tau \in \Sigma$ 
and each sort substitution $\theta$ over the sort parameters of $\tau$;

\item associates
a total function 
$(f{:}\theta(\tau_1)\cdots\theta(\tau_n)\theta(\tau))^{\str A}$ 
from $\theta(\tau_1)^{\str A} \times \cdots \times \theta(\tau_n)^{\str A}$ to
$\theta(\tau)^{\str A}$ to each function symbol $f{:}\tau_1\cdots\tau_n\tau \in \Sigma$ with $n > 0$ 
and each sort subsitution $\theta$ over the sort parameters of $\tau_1\cdots\tau_n\tau$.
\end{itemize}

If $\str B$ is an $\Omega$-structure with universe $B$ and 
$\Sigma$ is a subsignature of $\Omega$,
the \define{reduct} of $\str B$ to $\Sigma$ is the (unique) 
$\Sigma$-structure with universe $B$
that interprets its sort and function symbols exactly as $\str B$.
A structure $\str B$ is an \define{expansion} of a $\Sigma$-structure $\str A$
if $\str A$ is the $\Sigma$-reduct of $\str B$.
\qed
\end{definition}

Note that, as a consequence of overloading,
a $\Sigma$-structure does not interpret plain function symbols
but ranked function symbols.
Also note that 
any $\Sigma$-structure is also a $\Sigma'$-structure 
for every variant $\Sigma'$ of $\Sigma$.


\begin{definition}[Absolutely free structure]
Let $\str A$ be a $\Sigma$-structure with universe $A$ and 
let $G \subseteq A$.
Let $\Sigma_G$ be the expansion of $\Sigma$ obtained by adding to $\Sigma$
a constant symbol $a$ of sort $\sigma$
for every $a \in G$ and monomorphic sort $\sigma \in \sorts\Sigma$ 
such that $a \in \sigma^{\str A}$.
Then, $\str A$ is an \define{absolutely free structure (with generators $G$)}
if 
\begin{itemize}
\item
for all $\sigma \in \sortt{\Sigma}$, $\sigma^{\str A}$ is the set 
of well-sorted ground terms of signature $\Sigma_G$;

\item
$\str A$ interprets every function symbol 
$f{:}\sigma_1 \cdots \sigma_n \sigma \in \Sigma_G$ as the function that maps each tuple 
$(t_1,\ldots,t_n) \in \sigma_1^{\str A} \times \ldots \times \sigma_n^{\str A}$ 
to the term $f(t_1,\ldots,t_n)$.
\qed
\end{itemize}
\end{definition}

\pfrem{We might have to be careful in the above item, because functions are 
actually polymorphic and we might have to consider all instances.}

Intuitively, an absolutely free $\Sigma$-structure with a set $G$ of generators
interprets every well-sorted ground $\Sigma_G$-term as itself.
Note that the choice of generators affects the property of being absolutely free.
For instance, no structure without constant symbols can be absolutely free 
with an empty set of generators.
\medskip

The SMT-LIB logic considers only structures that interpret in a special way 
the sort \bool\ and any constructor, selector, and tester symbols in their signature.

\begin{definition}[SMT-LIB $\Sigma$-structure]
\label{def:structure}
An SMT-LIB structure is a $\Sigma$-structure $\str A$ such that

\begin{enumerate}
\item $\bool^{\str A} = \bo = \{ \false, \true \}$ with \false\ and \true\ distinct;

%CT unnecessary
%\item
%${\eqs}{:}\tau\tau\,\bool$ is interpreted as the identity predicate over $\tau^{\str A}$ \footnote{%
%That is,
%for all $\tau \in \sortt{\Sigma}$ and all $a,b \in \tau^{\str A}$,\ 
%${\eqs}^{\str A}(a,b) = \true$ iff $a$ is the same as $b$.
%};

%CT previous, incomplete restrictions on constructors
%\item
%for all (ranked) constructors $c{:}\tau_1\cdots\tau_n\tau \in \Sigma$ with $n>0$,
%$c^{\str A}$ is injective;
%
%\item
%for all sorts $\tau$ with a set $\{c_1, \ldots, c_n\}$ of $n>0$ constructors,
%the ranges of $c_1^{\str A}, \ldots, c_n^{\str A}$ form a partition of $\tau^{\str A}$;

\item
if $\Omega$ is the signature obtained from $\Sigma$ by removing 
all of its non-constructor function symbols,
the $\Omega$-reduct of $\str A$ is an absolutely free algebra with 
generators $\bigcup_{ \tau \in S} \tau^{\str A}$
where $S$ collects the sorts of $\Sigma$ that are not datatypes;

\item
for all constructors $c{:}\tau_1\cdots\tau_n\tau \in \Sigma$ with $n>0$,
selectors $g_1{:}\tau\tau_1, \ldots, g_n{:}\tau\tau_n$ 
with $\mathrm{sel}_\Sigma(c) = g_1\cdots g_n$, 
values $(v_1, \ldots, v_n) \in \tau_1^{\str A} \times \cdots \times \tau_n^{\str A}$,
and
$i=1,\ldots,n$,
\[
 g_i^{\str A}(c^{\str A}(v_1, \ldots, v_n)) = v_i\ ;
\]

\item
for all constructors $c{:}\bar{\tau}\tau \in \Sigma$,
testers $q$ with $\mathrm{tes}_\Sigma(c) = q$, and 
values $v \in \tau^{\str A}$,
\begin{center}
 $q^{\str A}(v) = \true$ iff $v$ is in the range of $c^{\str A}$.
\end{center}
\end{enumerate}
\end{definition}


From now on, when we say ``structure'' we will mean ``SMT-LIB structure.''

\begin{remark}
The restrictions in SMT-LIB structures on the interpretation 
of constructors, selectors and testers,
together with the well-foundedness restrictions on those constructs in signatures,
guarantee that sorts with constructors indeed denote algebraic datatypes
as traditionally understood in the literature.
\end{remark}

\begin{remark}[Partiality of selectors]
As in classical first-order logic, all function symbols in a signature $\Sigma$ 
are interpreted as total functions in a $\Sigma$-structure $\str A$.  
This means in particular that if $g{:}\tau\tau_i \in \Sigma$ is a selector,
the function $g^{\str A}$ returns a value even for inputs
outside the range of $g$'s constructor.
Definition~\ref{def:structure} imposes no constraints on that value,
other than it belongs to $\tau_i^{\str A}$.
For instance, in a structure $\str A$ with a sort for integer lists with constructors
$\mathrm{nil}$ and $\mathrm{insert}$ and 
selectors $\mathrm{head}$ and $\mathrm{tail}$ for $\mathrm{insert}$,
the function $\mathrm{head}^{\str A}$ maps $\mathrm{nil}^{\str A}$ 
to \emph{some} integer value.
Similarly,
$\mathrm{tail}^{\str A}$ maps $\mathrm{nil}^{\str A}$ to \emph{some} integer list.
This is consistent with the general modeling of partial functions 
in SMT-LIB as underspecified total functions---which requires a solver 
to consider all possible (well-sorted) ways to make a partial function total.
\end{remark}


The notion of isomorphism between structures introduced below 
is needed for Definition~\ref{def:theorycomb}, Theory Combination,
in Section~\ref{sec:theories}.

\begin{definition}[Isomorphism]
Let $\str A$ and $\str B$ be two $\Sigma$-structures 
with respective universes $A$ and $B$.
A mapping $h:A \to B$ is a \define{homomorphism} from $\str A$ to $\str B$
if
\begin{enumerate}
\item
for all $\tau \in \sortt{\Sigma}$ and $a \in \tau^{\str A}$,
\[
 h(a) \in  \tau^{\str B}\ ;
\]
\item
for all $f{:}\tau_1\ldots\tau_n\tau \in \Sigma$ with $n > 0$
and 
\(
 (a_1, \ldots, a_n) \in 
 \tau_1^{\str A} \times \cdots \times \tau_n^{\str A} ,
\)
\[
 h((f{:}\tau_1\ldots\tau_n\tau)^{\str A}(a_1, \ldots, a_n)) = 
 (f{:}\tau_1\ldots\tau_n\tau)^{\str B}(h(a_1), \ldots, h(a_n)) \ . 
\]
\end{enumerate}
%
A homomorphism between $\str A$ and $\str B$ is 
an \define{isomorphism} of $\str A$ onto $\str B$
if it is invertible and its inverse is a homomorphism from $\str B$ to $\str A$. 
\qed
\end{definition}

Two $\Sigma$-structures $\str A$ and $\str B$ are \define{isomorphic}
if there is an isomorphism from one onto the other.
Isomorphic structures are interchangeable for satisfiability purposes
because one satisfies a set of $\Sigma$-sentences if and only if the other one does.



\subsection{The meaning of terms}

A \define{valuation} into a $\Sigma$-structure $\str A$ is 
a partial mapping $v$ from $\va \times \sortt{\Sigma}$ to the set 
of all domain elements of $\str A$ such that,
for all $x \in \va$ and $\tau \in \sortt{\Sigma}$, 
$v(x{:}\tau) \in \tau^{\str A}$.
%
We denote by $v[x_1{:}\tau_1 \mapsto a_1,\;\ldots,\;x_n{:}\tau_n \mapsto a_n]$
the valuation that maps $x_i{:}\tau_i$ to $a_i \in \tau_i^{\str A}$ 
for $i=1,\ldots,n$ and is otherwise identical to $v$.
%Note that 
%$v[x_1 \mapsto a_1, \ldots, x_n \mapsto a_n]$ need not be 
%a $\Sigma$-valuation\footnote{%
%Because each $a_i$ need not be in $\tau_i^{\str A}$ 
%where $\tau_i$ is the sort of $x_i$ in $\Sigma$.
%}
%but it is definitely a $\Sigma'$-valuation 
%for some variant $\Sigma'$ of $\Sigma$.\endnote{
%This is technicality necessary to make 
%the function $\mea{\_}{\_}$ well-defined.
%It is a consequence of the fact that 
%while all variables are assigned a sort a signature $\Sigma$,
%quantifiers in a $\Sigma$-formulas can freely change 
%the sort of their bound variables.
%This means, for instance that
%if $\exists x{:}\tau \varphi$ is a formula of some signature $\Sigma$,
%its subformula $\varphi$ is, for typing \emph{and} semantic purposes, 
%a formula of signature $\Sigma[x{:}\tau]$.
%}

If $v$ is a valuation into $\Sigma$-structure $\str A$,
the pair $\inter{I} = (\str A, v)$ is a \define{$\Sigma$-interpretation}.
We write $\inter{I}[x_1{:}\tau_1 \mapsto a_1,\;\ldots,\;x_n{:}\tau_n \mapsto a_n]$
as an abbreviation for the $\Sigma'$-interpretation
\[
 (\str A', v[x_1{:}\tau_1 \mapsto a_1,\;\ldots,\; x_n{:}\tau_n \mapsto a_n])
\]
where $\Sigma' = \Sigma[x_1{:}\tau_1,\;\ldots,\; x_n{:}\tau_n]$
and
$\str A'$ is just $\str A$ but seen as a $\Sigma'$-structure.

A $\Sigma$-interpretation $\inter{I}$ assigns a meaning 
to well-sorted $\Sigma$-terms by means of a uniquely determined
(total) mapping $\mea{\inter I}{\_}$ of such terms into the universe 
of its structure.

\begin{definition}
Let $\Sigma$ be an SMT-LIB signature
and 
let $\inter{I}$ be a $\Sigma$-interpretation.
For every well-sorted term $t$ of sort $\tau$ with respect to $\Sigma$,
$\mea{\inter I}{t}$ is defined recursively as follows,
where Rule~\ref{ruleI} applies to polymorphic terms
and the rest of the rules apply only to monomorphic terms.

\begin{enumerate}
\item \label{ruleI}
$\mea{\inter I}{t} = \true$ 
iff \ $\mea{\inter I}{\theta(t)} = \true$ 
for all $\theta$ \ such that \ 
\(
\begin{cases}
 \theta = \{u_1 \mapsto \sigma_1,\;\ldots,\;u_n\mapsto \sigma_n\}, \\
 \{u_1, \ldots u_n\} = pars(t),\; n > 0\\
 \sigma_1,\ldots,\sigma_n \in \sortt{\Sigma}
\end{cases}
\)

\item
$\mea{\inter I}{t\;\alpha_1\;\cdots\;\alpha_n} = \mea{\inter I}{t}$

\item
$\mea{\inter I}{x} = v(x{:}\sigma)$ 
\quad if $\inter I = (\str A, v)$  

\item
$\mea{\inter I}{\hat{f}\;t_1\;\ldots\;t_n} = 
 (f{:}\sigma_1\cdots\sigma_n\sigma)^{\str A}(a_1,\ldots,a_n)
$ 
\quad if \  
\(
\begin{cases}
 \inter I = (\str A, v) \text{ with signature } \Sigma,\\
 \hat{f} = f \text{ or } \hat{f} = f^\sigma,\\
 \text{for } i=1,\ldots,n \\
 \quad \Sigma \vdash t_i : \sigma_i \text{ and } a_i = \mea{\inter I}{t_i}
\end{cases}
\)

\item
$\mea{\inter I}{\akey{let}\ x_1 = t_1\;\cdots\; x_n = t_n\ \akey{in}\ t} = 
 \mea{\inter I'}{t}
$ 
\ if \  
\(
\begin{cases}
 \inter{I} \text{ has signature } \Sigma, \\
 \text{for } i=1,\ldots,n \\
 \quad \Sigma \vdash t_i : \sigma_i \text{ and } a_i = \mea{\inter I}{t_i}, \\
 \inter{I}' = {\inter I}[x_1{:}\sigma_1\mapsto a_1,\;\ldots,\;x_n{:}\sigma_n\mapsto a_n]
\end{cases}
\)

\item
$\mea{\inter I}{\exists\, x_1{:}\sigma_1\, \cdots\, x_n{:}\sigma_n\; t} = \true$ 
iff \ $\mea{\inter I'}{t} = \true$ 
for some $\inter I'$

\qquad such that
\(
\begin{cases}
 \inter{I}'= {\inter I}[x_1{:}\sigma_1\mapsto a_1,\;\ldots,\;x_n{:}\sigma_n\mapsto a_n],\\
 (a_1,\ldots,a_n) \in \sigma_1^{\str A} \times \cdots \times \sigma_n^{\str A},\\
 \inter I = (\str A, v),\\
\end{cases}
\)

\item
$\mea{\inter I}{\forall\, x_1{:}\sigma_1\, \cdots\, x_n{:}\sigma_n\; t} = \true$ 
iff \ $\mea{\inter I'}{t} = \true$ 
for all $\inter I'$

\qquad such that
\(
\begin{cases}
 \inter{I}'= {\inter I}[x_1{:}\sigma_1\mapsto a_1,\;\ldots,\;x_n{:}\sigma_n\mapsto a_n],\\
 (a_1,\ldots,a_n) \in \sigma_1^{\str A} \times \cdots \times \sigma_n^{\str A},\\
 \inter I = (\str A, v),\\
\end{cases}
\)

\item
\(
 \mea{\inter I}{\akey{match}\ t\ \akey{with}\ c \to t_0\ \:p_1 \to t_1 \;\cdots\; p_n \to t_n} = \mea{\inter I}{t_0}
\)

\qquad if \  
\(
\begin{cases}
 \inter I = (\str A, v),\\
 \mea{\inter I}{t} \text{ is in the range of } c^{\str A}
\end{cases}
\)

\item
\(
 \mea{\inter I}{\akey{match}\ t\ \akey{with}\ (c\: x_1 \cdots x_{k+1}) \to t_0\ \:p_1 \to t_1 \;\cdots\; p_n \to t_n} =
\)

\(
 \mea{\inter I}{\akey{let}\ x_1 = g_1(t) \;\cdots\; x_{k+1} = g_{k+1}(t)\ \akey{in}\ t_0}
\)

\qquad if \  
\(
\begin{cases}
 \inter I = (\str A, v) \text{ with signature } \Sigma,\\
 \mea{\inter I}{t} \text{ is in the range of } c^{\str A}, \\
 \mathrm{sel}_\Sigma(c) = g_1\cdots g_{k+1}
\end{cases}
\)

\item
\(
 \mea{\inter I}{\akey{match}\ t\ \akey{with}\ (c\: \bar x) \to t_0\ \:p_1 \to t_1 \;\cdots\; p_n \to t_n} =
\)

\(
 \mea{\inter I}{\akey{match}\ t\ \akey{with}\ \:p_1 \to t_1 \;\cdots\; p_n \to t_n\ (c\: \bar x) \to t_0}
\)

\qquad if \ 
\(
\begin{cases}
 \inter I = (\str A, v),\\
 \mea{\inter I}{t} \text{ is not in the range of } c^{\str A}
\end{cases}
\)

\item
\(
 \mea{\inter I}{\akey{match}\ t\ \akey{with}\ x_0 \to t_0\ \:p_1 \to t_1 \;\cdots\; p_n \to t_n} = 
 \mea{\inter I}{\akey{let}\ x_0 = t\ \akey{in}\ t_0}
\)
\qed
\end{enumerate}
\end{definition}

\noindent 
One can show that $\mea{\inter I}{\_}$ is well-defined, and hence total, over
\emph{closed} terms that are \emph{well-sorted} with respect to $\inter I$'s signature.


A $\Sigma$-interpretation $\inter I$ \define{satisfies}
a $\Sigma$-formula $\varphi$ if $\mea{\inter I}{\varphi} = \true$,
and \define{falsifies} it if $\mea{\inter I}{\varphi} = \false$.
The formula $\varphi$ is \define{satisfiable} 
if there is a $\Sigma$-interpretation $\inter I$ that satisfies it,
and is \define{unsatisfiable} otherwise.

For a closed term $t$,
its meaning $\mea{\inter I}{t}$ 
in an interpretation $\inter{I} = (\str A, v)$ is independent of the choice 
of the valuation $v$---because the term has no free variables.
For such terms then, 
we can write $\mea{\str A}{t}$ instead of $\mea{\inter I}{t}$.
Similarly,
for sentences, we can speak directly of a \emph{structure} 
satisfying or falsifying the sentence.
A $\Sigma$-structure that satisfies a sentence is also called 
a \define{model} of the sentence.  



%-------------------------------------------------------------------------------
\section{Theories} \label{sec:theories}
%-------------------------------------------------------------------------------

%In SMT, one is not usually interested in arbitrary models 
%of sentences
%but models belonging to a given \emph{theory}
%$\T$ constraining the interpretation of the symbols of $\Sigma$.
Theories are traditionally defined as sets of sentences.
Alternatively, and more generally, 
in SMT-LIB a theory is defined as a class of structures with the same signature.

\begin{definition}[Theory]
For any signature $\Sigma$,
a \define{$\Sigma$-theory} is a class of $\Sigma$-structures.
Each of these structures is a \define{model} of the theory.
\end{definition}

Typical SMT-LIB theories consist of a single model 
(e.g., the integers)
or 
of the class of all structures that satisfy some set of sentences---the
\define{axioms} of the theory.
Note that in SMT-LIB there is no requirement that the axiom set be
finite or even recursive.

%A $\Sigma$-theory is \define{axiomatizable} 
%if it consists of all the models of some recursive set of $\Sigma$-sentences.
%Note that in SMT-LIB, 
%background theories are not restricted to be axiomatizable.

\subsection{Combined Theories} \label{sec:comb-theories}

SMT-LIB uses both \emph{basic} theories, 
obtained as instances of a theory declaration schema,
and \emph{combined} theories, obtained by combining together
suitable instances of different theory sche\-mas.
The combination mechanism is defined below.

Two signatures $\Sigma_1$ and $\Sigma_2$ are \define{compatible} 
if they have the same sort symbols, 
have the same datatype constructors, selectors and testers, and they agree 
on both the arity they assign to sort symbols and 
the sorts they assign to variables.\footnote{%
Observe that compatibility is an equivalence relation on signatures.
}
Two theories are \define{compatible} if they have compatible signatures.
The \define{combination} $\Sigma_1 + \Sigma_2$
of two compatible signatures $\Sigma_1$ and $\Sigma_2$
is the smallest compatible signature
that is an expansion of both $\Sigma_1$ and $\Sigma_2$,
i.e., the unique signature $\Sigma$ compatible 
with $\Sigma_1$ and $\Sigma_2$ such that,
for all $f \in \fu$ and $\bar{\tau} \in \sortt{\Sigma}^+$,
$f{:}\bar{\tau} \in \Sigma$ iff 
$f{:}\bar{\tau} \in \Sigma_1$ or
$f{:}\bar{\tau} \in \Sigma_2$.

\begin{definition}[Theory Combination]
\label{def:theorycomb}
Let $\T_1$ and $\T_2$ be two theories 
with compatible signatures $\Sigma_1$ and $\Sigma_2$, respectively. 
The \define{combination} $\T_1 + \T_2$ of $\T_1$ and $\T_2$
consists of all $(\Sigma_1+\Sigma_2)$-structures
whose reduct to $\Sigma_i$ is isomorphic to a model of $\T_i$,
for $i=1,2$.
\qed
\end{definition}

Over pairwise compatible signatures
the signature combination operation $+$ is associative and commutative.
The same is also true for the theory combination operation $+$ over
compatible theories.
This induces, for every $n > 1$,
a unique $n$-ary combination $\T_1 + \cdots + \T_n$ 
of mutually compatible theories $\T_1,\ldots,\T_n$
in terms of nested binary combinations.
\define{Combined} theories in SMT-LIB are exclusively
theories of the form $\T_1 + \cdots + \T_n$ 
for some basic SMT-LIB theories $\T_1,\ldots,\T_n$.

SMT is about checking the satisfiability or the entailment 
of formulas \emph{modulo} some (possibly combined) theory $\T$.
This standard adopts the following precise formulation of such notions.

\begin{definition}[Satisfiability and Entailment Modulo a Theory]
For any $\Sigma$-theory $\T$,
a $\Sigma$-sentence is \define{satisfiable in $\T$}
iff it is satisfied by one of $\T$'s models.
A set $\Gamma$ of $\Sigma$-sentences
\define{$\T$-entails} a $\Sigma$-sentence $\varphi$,
written $\Gamma \models_\T \varphi$,
iff every model of $\T$ that satisfies all sentences in $\Gamma$
satisfies $\varphi$ as well.
\end{definition}


\subsection{Theory declarations} \label{sec:theory-decl}

\begin{figure}[t]
\theories
\caption{Abstract syntax for theory declarations}
\label{fig:theory-declaration}
\end{figure}

In SMT-LIB, basic theories are obtained as instances of theory declarations.
(In contrast, combined theories are defined in logic declarations.)
An abstract syntax of theory declarations is defined in 
Figure~\ref{fig:theory-declaration}.
The symbol $\Pi$ in parametric function symbol declarations 
is a (universal) binder for the sort parameters---and corresponds 
to the symbol \ter{par} in the concrete syntax.

To simplify the meta-notation let $T$ denote 
a theory declaration with theory name $T$.
Given such a theory declaration,
assume first that $T$ has no \akey{sorts-description} and
\akey{funs-description} attributes,
and
let $S$ and $F$ be respectively the set of all sort symbols and 
all function symbols occurring in $T$.
Let $\Omega$ be a signature
\emph{whose set of sort symbols that are not datatypes
includes all the symbols in $S$, with the same arity}.
The definition provided in the \akey{definition} attribute of $T$
must be such that
every signature like $\Omega$ above uniquely determines 
a theory $\widehat T = T[\Omega]$ as an instance of $T$
with signature $\widehat\Omega$ defined as follows:
\begin{enumerate}
\item
$\sorts{\widehat\Omega} = \sorts{\Omega}$
and
$\funs{\widehat\Omega} = F \cup \funs{\Omega}$,

\item
no variables are sorted in $\widehat\Omega$,\endnote{
This requirement is for concreteness.
Again, since we work with closed formulas, 
which internally assign sorts to their variables,
the sorting of variables in a signature is irrelevant.
}

\item
for all $f \in \funs{\widehat\Omega}$ and 
$\bar{\tau} \in (\sorts{\widehat\Omega})^+$, 

$f{:}\bar{\tau} \in \widehat\Omega$ iff
\begin{enumerate}
\item
$f{:}\bar{\tau} \in \Omega$, or

\item \label{decl-i}
$T$ contains a declaration of the form $f\;\bar{\tau}\;\bar{\alpha}$, or

\item \label{decl-ii}
$T$ contains a declaration of the form $\Pi\;\bar{u}\;(f\;\bar{\tau}\;\bar{\alpha})$ and
$\bar{\tau}$ is an instance of $\bar \tau$.
\end{enumerate}

\item
A function symbol of $\widehat\Omega$ is a datatype constructor/selector/tester 
in $\widehat\Omega$ iff it is so in $\Omega$.
\end{enumerate}

We say that a ranked function symbol $f{:}\bar{\tau}$ of $\widehat\Omega$ 
is \define{declared in $T$} 
if $f{:}\bar{\tau} \in \widehat\Omega$
because of Point~\ref{decl-i} or \ref{decl-ii} above.
The \define{free sort symbols} of $\widehat T$ are
the sort symbols of $\widehat\Omega$ that are not in $S$
and are not datatypes.
Similarly, the \define{free function symbols} of $\widehat T$
are the ranked function symbols of $\widehat\Omega$
that are not declared in $T$
and are not datatype constructors, selectors or testers.\footnote{%
Note that because of overloading 
we talk about \emph{ranked} function symbols being free or not,
not just function symbols.
}
This terminology is justified by the following additional requirement on $T$.

The definition of $T$ must be \emph{parametric}, in this sense: 
it must not constrain the free symbols of 
any instance $T[\Omega]$ of $T$ in any way.
Technically, $T$ must be defined so that
the set of models of $T[\Omega]$ is closed 
under any changes in the interpretation of the free symbols.
That is,
every structure obtained from a model of $T[\Omega]$
by changing only the interpretation of $T[\Omega]$'s free symbols should be 
a model of $T[\Omega]$ as well.\endnote{
Admittedly,
this requirement on theory declarations is somewhat hand-wavy.
Unfortunately, it is not possible to make it a lot more rigorous
because a theory declaration can use natural language 
to define its class of instance theories.
The point is again that 
the definition of the class should impose no constraints on 
the interpretation of free sort symbols and free function symbols.
}


The case of theory declarations with \akey{sorts-description} and
\akey{funs-description} attributes is similar.


%-------------------------------------------------------------------------------
\section{Logics} \label{sec:logics}
%-------------------------------------------------------------------------------

\begin{figure}[t]
\logics
\caption{Abstract syntax for logic declarations}
\label{fig:logic-declaration}
\end{figure}


A logic in SMT-LIB is any sublogic of the main SMT-LIB logic obtained by 
\begin{itemize}
\item
fixing a signature $\Sigma$ and a $\Sigma$-theory $\T$,

\item
restricting the set of structures to the models of $\T$,
and 

\item
restricting the set of sentences to some subset of 
the set of all $\Sigma$-sentences.
\end{itemize}

A \define{model} of a logic with theory $\T$ is any model of $\T$;
a sentence is \define{satisfiable} in the logic iff it is satisfiable in $\T$.



\subsection{Logic declarations} \label{sec:logic-decl}

Logics are specified by means of logic declarations.
Contrary to the theory declarations, 
a logic declaration specifies a \emph{single} logic, 
not a class of them,
so we call the logic $L$ too.
An abstract syntax of logic declarations is defined in 
Figure~\ref{fig:logic-declaration}.

Let $L$ be a logic declaration whose \akey{theories} attribute has value
$T_1, \ldots, T_n$.

\paragraph{Theory.}
The logic's theory is the theory $\T$ uniquely determined as follows.
For each $i=1,\ldots,n$,
let $S_i$ be the set of all sort symbols occurring in $T_i$.
The text in the \akey{language} attribute of $L$ may specify
an additional set $S_0$ of sort symbols and 
an additional set of ranked function symbols with ranks over $\sortt{S}^+$
where $S = \bigcup_{i=0,\ldots,n} S_i$.
Let $\Omega$ be the smallest signature with $\sorts{\Omega} = S$
containing all those ranked function symbols.
Then for each $i=1,\ldots,n$, 
let $T_i[\Omega]$ be the instance of $T_i$ determined by $\Omega$ 
as described in Subsection~\ref{sec:theory-decl}.
The theory of $L$ is 
\[
 \T = T_1[\Omega] + \cdots + T_n[\Omega]\ .
\]

Note that $\T$ is well defined.
To start, $\Omega$ is well defined 
because any sort symbols shared by two declarations among
$T_1, \ldots, T_n$ have the same arity in them.
The theories $T_1[\Omega], \ldots, T_n[\Omega]$ are well defined 
because $\Omega$ satisfies the requirements in Subsection~\ref{sec:theory-decl}.
Finally, 
the signatures of $T_1[\Omega], \ldots, T_n[\Omega]$ are pairwise compatible
because they all have the same sort symbols, 
each with the same arity in all of them.

 

\paragraph{Values.}
The $\akey{values}$ attribute is expected to designate 
for each sort $\tau$ of the logic's theory $\T$ 
a distinguished set $V_\tau$ of ground terms called \define{values}.  
The definition of $V_\tau$ should be such that 
every sentence satisfiable in the logic $L$ is satisfiable
in a model $\str A$ of $\T$ 
where each element of $\tau^{\str A}$
is denoted by some element of $V_\tau$.
In other words, if $\Sigma$ is $\T$'s signature,
$\str A$ is such that,
for all $\tau \in \sorts{\Sigma}$ and 
all $a \in \tau^{\str A}$,
$a = \mea{\inter A}{t}$ for some $t \in V_\tau$.
For example,
in a logic of the integers,
the set of values for the integer sort
might consist of all the terms of the form $0$ or $[-]n$
where $n$ is a non-zero numeral.

For flexibility, we do not require that $V_\tau$ be minimal.
That is, it is possible for two terms of $V_\tau$ to denote 
the same element of $\tau^{\str A}$.
For example,
in a logic of rational numbers, the set of values for the rational sort
might consist of all the terms of the form $[-]m/n$
where $m$ is a numeral and $n$ is a non-zero numeral.
This set covers all the rationals 
but, in contrast with the previous example, is not minimal
because, for instance, $3/2$ and $9/6$ denote the same rational.

Note that the requirements on $V_\tau$ can be always trivially satisfied 
by $L$ by making sure that the signature
$\Omega$ above contains 
a distinguished set of infinitely many additional free constant symbols of 
sort $\tau$,
and defining $V_\tau$ to be that set.
We call these constant symbols \define{abstract values}.
Abstract values are useful to denote the elements of uninterpreted  
sorts or sorts standing for structured data types
such as lists, arrays, sets and so on.\footnote{%
The concrete syntax reserves a special format for constant symbols 
used as abstract values: 
they are members of the \nter{symbol} category 
that start with the character \ter{@}.
}

Recall that algebraic datatypes are not defined in theories but directly 
at the level of the underlying SMT-LIB logic. 
For each such sort, the set of values is fixed to the set 
of constructor terms built over values from other sorts.
For example, in a parametric list datatype $\delta$ 
with the usual $\mathrm{nil}$ and $\mathrm{cons}$ constructors,
the set of values for $\delta$ consists of $\mathrm{nil}$ and 
all terms of the form 
$\mathrm{cons}\ v_1\ (\mathrm{cons}\ v_2\ \cdots\ (\mathrm{cons}\ v_n\ \mathrm{nil})\ \cdots\ )$
where $n > 0$ and 
$v_1, \ldots, v_n$ are (recursively) values of the same sort.






